\section{related-works}
\label{sec:related-works}

\subsection{Clustering-based Approaches}
\label{subsec:related-works:clustering-based-approaches}

Clustering refers to techniques for grouping points in a way that provides value.
This is generally done by assigning \textit{similar} points to the same cluster.
Given a clustering and a new point, one can estimate the anomalousness of the new point by measuring its distance to its nearest cluster.

There have been few advancements in clustering techniques over the past decade~\cite{wang2019progress}.
This may be explained by the poor performance of clustering in high-dimensional space~\cite{zhang2013advancements} thus far.
The major approaches are as follows.

Distance-based clustering relies on some distance measure to partition data into some number of clusters.
Within this approach, the numbers and/or sizes of clusters are often predefined: either user-specified, or chosen at random~\cite{wang2019progress}.
Some examples of distance-based clustering are:
K-Means~\cite{macqueen1967some},
PAM~\cite{kaufman2009finding},
CLARANS~\cite{ng1994efficient} and
CLARA~\cite{kaufman2009finding}.

Hierarchical clustering methods utilize a tree-like structure, where points are allocated into leaf nodes~\cite{wang2019progress}.
These tree-like structures can be created bottom-up (agglomerative clustering) or top-down (divisive clustering)~\cite{agrawal1998automatic}.
A major drawback of these methods is the high cost of pairwise difference computations required to build each level of the tree.
Examples of hierarchical clustering include
MST~\cite{charles_zahn_graph_1971},
CURE~\cite{guha1998cure} and
CHAMELEON~\cite{karypis1999hierarchical}.

Density-based clustering methods rely on finding regions of high point-density separated by regions of low point-density.
These algorithms generally do not work well when data are sparse or uniformly distributed.
Some examples of density-based clustering algorithms are
DBSCAN~\cite{ester1996density} and
DENCLUE~\cite{hinneburg1998efficient}.

Grid-based clustering works via segmenting the entire space into a discrete number of cells, and then scanning those cells to find regions of high density.
Utilizing a grid structure for clustering means that these algorithms typically scale well to larger datasets.
Some examples of grid-based clustering include
STING~\cite{wang1997sting},
Wavecluster~\cite{sheikholeslami2000wavecluster}, and
CLIQUE~\cite{agrawal1998automatic}.

\subsection{Distanced-based Approaches}
\label{subsec:related-works:distanced-based-approaches}

Distance-based methods find anomalous points via distance comparisons.
These methods largely employ k-Nearest Neighbors as their substrate~\cite{wang2019progress}.
Distance-based approaches tend to use the following intuitions:
points with fewer than $p$ other points within some distance $d$ are outliers;
the $n$ points with the greatest distances to their $k^{th}-$nearest neighbor are outliers;
or the $n$ points with the greatest average distance to their $k$ nearest neighbors are outliers.

\subsection{CHAODA}
\label{subsec:related-works:chaoda}

In this paper we introduce a novel technique, Clustered Learning of Approximate Manifolds (CLAM).
This approach uses divisive hierarchical clustering to learn a manifold in a Banach space~\cite{banach1929fonctionnelles} defined by a distance metric.
In actuality, we do not require a metric.
The space may be defined by a distance \textit{function} that does not obey the triangle inequality, though this is not always optimal.
Given a learned approximate manifold, we can almost trivially implement several anomaly-detection algorithms.
In this manuscript, we present a collection of five such algorithms implemented on CLAM: CHAODA (Clustered Hierarchical Anomaly and Outlier Detection Algorithms).

The manifold learning component is derived from prior work, CHESS~\cite{ishaq2019entropy}, to accelerate approximate search on large high-dimensional datasets.
CLAM begins by divisively clustering the data until each cluster contains only one datum.
CLAM then delineates \textit{layers} of clusters at each depth in the tree.
Each layer comprises all clusters that would have been leaf nodes if the tree building had been halted at the given depth.

CLAM then builds a graph for each layer in the tree by creating edges between clusters that have overlapping volumes.
This process effectively learns the manifold on which the data lie at various resolutions, given by the depth of the layer.
This is analogous to a ``filtration'' in computational topology~\cite{carlsson2009topology}.
Once we have learned a manifold, we can ask about the cardinality of various clusters at different depths, how connected a given cluster is, or even how often a cluster is visited by random walks on the manifold.

We test our methods on 24 real-world datasets.
The datasets span a wide variety of domains, each having a different quantity of anomalous data.
We consider several different definitions of outliers and anomalies: \textbf{distance-based}, examining several classical distance-based definitions of outliers, relying on CLAM's use of distance to cluster data; \textbf{density-based}, examining the cardinality of clusters, under the hypothesis that clusters with lower cardinality are more likely to contain outliers; \textbf{graph-based}, examining several graph-theoretic methods for anomaly detection, given graphs constructed from layers of clusters.

Historically, clustering approaches have suffered from several problems.
The most common deficiencies are: the effective treatment of high dimensionality, the ability to interpret results, and the ability to scale to exponentially-growing datasets ~\cite{agrawal1998automatic}.
CLAM largely resolves these problems.


\section{Introduction}
\label{sec:introduction}

TODO

%Detecting anomalies and outliers from data is a well-studied problem in machine learning.
%When data occupy easily-described distributions, such as the Gaussian, the task is relatively easy: one need only identify when a datum is sufficiently far from the mean.
%However, in ``big data'' scenarios, where data can occupy high-dimensional spaces, anomalous behavior becomes harder to quantify.
%If the data happen to be uniformly distributed, one can conceive of simple mechanisms, such as a one-class SVM, that would be effective in any number of dimensions.
%However, real-world data are rarely distributed uniformly.
%Instead, data often obey the ``manifold hypothesis''~\cite{fefferman2016testing}, occupying a low-dimensional manifold in a high-dimensional embedding space.
%This low-dimensional manifold may weave itself through the high-dimensional space much like a crumpled sheet of paper does in 3-dimensional space.
%Detecting anomalies in such a landscape is not easy;
%in particular, correctly identifying an anomalous datum that sits within the gaps of a lower-dimensional manifold presents a challenge.
%
%Anomalies (data that do not belong to a distribution) and outliers (data which represent extrema of a distribution) can arise from many sources:
%errors in measurement or collection of data;
%novel, previously-unseen instances of data;
%normal behavior evolving into abnormal behavior;
%and adversarial attacks as inputs to machine-learning algorithms~\cite{elsayed2018adversarial}.
%Modern algorithms designed to detect anomalous behavior fail for a variety of reasons, in particular when anomalies live close to, but not on, a complex manifold in high-dimensional space.
%Our approach is designed to learn these complex manifolds.
%Here we briefly survey contemporary approaches to anomaly detection in order to provide the context needed to understand how our approach differs.

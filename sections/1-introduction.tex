\section{Introduction}
\label{sec:introduction}

Detecting anomalies and outliers from data is a well-studied problem in machine learning.
When data occupy easily-characterizable distributions, such as the Gaussian, the task is relatively easy: one need only identify when a datum is sufficiently far from the mean.
However, in ``big data'' scenarios, where data can occupy high-dimensional spaces, anomalous behavior becomes harder to quantify.
If the data happen to be uniformly distributed, one can conceive of simple mechanisms, such as a one-class SVM, that would be effective in any number of dimensions.
However, real-world data are rarely distributed uniformly.
Instead, data often obey the ``manifold hypothesis''~\cite{fefferman2016testing}, occupying a low-dimensional manifold in a high-dimensional embedding space.
This low-dimensional manifold may weave itself through the high-dimensional space much like a crumpled sheet of paper does in 3-dimensional space.
Detecting anomalies in such a landscape is not easy;
in particular, correctly identifying an anomalous datum that sits within the gaps of a lower-dimensional manifold presents a challenge.

Anomalies (data that do not belong to a distribution) and outliers (data which represent the extrema of a distribution) can arise from many sources:
\begin{itemize}
    \item errors in measurement or collection of data,
    \item novel, previously-unseen instances of data,
    \item normal behavior evolving into abnormal behavior
    \item and adversarial attacks as inputs to machine-learning algorithms~\cite{elsayed2018adversarial}.
\end{itemize}

Modern algorithms designed to detect anomalous behavior fail for a variety of reasons, in particular when anomalies live close to, but not on, a complex manifold in high-dimensional space.
Our approach is designed to map these complex manifolds.
Here we briefly survey contemporary approaches to anomaly detection in order to provide the context needed to understand how our approach differs.

\subsection{Clustering-based Approaches}
\label{subsec:introduction:clustering-based-approaches}

Clustering refers to techniques for grouping points in a way that provides value.
This is generally done by assigning \textit{similar} points to the same cluster.
Given a clustering and a new point, one can estimate the anomalousness of the new point by measuring its distance to its nearest cluster.

There have been few advancements in clustering techniques over the past decade~\cite{wang2019progress}.
This may be explained by the poor performance of clustering in high-dimensional space~\cite{zhang2013advancements} thus far.
The major approaches are as follows.

\subsubsection{Distance-based Clustering}
\label{subsubsec:introduction:clustering-based-approaches:distance-based-clustering}
relies on some distance measure to partition data into some number of clusters.
Within this approach, the numbers and/or sizes of clusters are often predetermined: either user-specified, or chosen at random~\cite{wang2019progress}.
Some examples of distance-based clustering are:
K-Means~\cite{macqueen1967some},
PAM~\cite{kaufman2009finding},
CLARANS~\cite{ng1994efficient} and
CLARA~\cite{kaufman2009finding}.

\subsubsection{Hierarchical Clustering}
\label{subsubsec:introduction:clustering-based-approaches:hierarchical-clustering}
methods utilize a tree-like structure, where points are allocated into leaf nodes~\cite{wang2019progress}.
These tree-like structures can be created bottom-up (agglomerative clustering) or top-down (divisive clustering)~\cite{agrawal1998automatic}.
A major drawback of these methods is the high cost of pairwise difference computations required to build each level of the tree.
Examples of hierarchical clustering include
MST~\cite{charles_zahn_graph_1971},
CURE~\cite{guha1998cure} and
CHAMELEON~\cite{karypis1999hierarchical}.

\subsubsection{Density-based Clustering}
\label{subsubsec:introduction:clustering-based-approaches:density-based-clustering}
methods rely on finding regions of high point-density separated by regions of low point-density.
These algorithms generally do not work well when data are sparse or uniformly distributed.
Some examples of density-based clustering algorithms are
DBSCAN~\cite{ester1996density} and
DENCLUE~\cite{hinneburg1998efficient}.

\subsubsection{Grid-based Clustering}
\label{subsubsec:introduction:clustering-based-approaches:grid-based-clustering}
works via segmenting the entire space into a discrete number of cells, and then scanning those cells to find regions of high density.
Utilizing a grid structure for clustering means that these algorithms typically scale well to larger datasets.
Some examples of grid-based clustering include
STING~\cite{wang1997sting},
Wavecluster~\cite{sheikholeslami2000wavecluster}, and
CLIQUE~\cite{agrawal1998automatic}.

In this paper, we compare against the following Clustering-based approaches:
\begin{itemize}
    \item Clustering Based Local Outlier Factor (CBLOF)~\cite{he2003cblof}.
\end{itemize} 


\subsection{Graph-based Approaches}
\label{subsec:introduction:graph-based-approaches}

TODO

In this paper, we compare against the following Graph-based approaches:
\begin{itemize}
    \item Connectivity-based Outlier Factor (COF)~\cite{tang2002cof}
\end{itemize}


\subsection{Distanced-based Approaches}
\label{subsec:related-works:distanced-based-approaches}

Distance-based methods find anomalous points via distance comparisons.
These methods largely employ k-Nearest Neighbors as their substrate~\cite{wang2019progress}.
Distance-based approaches tend to use the following intuitions:
points with fewer than $p$ other points within some distance $d$ are outliers;
the $n$ points with the greatest distances to their $k^{th}-$nearest neighbor are outliers;
or the $n$ points with the greatest average distance to their $k$ nearest neighbors are outliers.

In this paper, we compare to Connectivity-Based Outlier Factor (COF)\cite{tang2002cof}, Histogram-Based Outlier Detection (HBOS)\cite{goldstein2012hbos}, Isolation-Forest Outlier Detector (IFOREST)~\cite{tony2008iforest,tony2012iforest}, k-Nearest Neighbors (kNN)~\cite{ramaswamy2000efficient, sridhar2000knn, fabrizio2002knn}, Linear Model Deviation-base outlier Detection(LMDD)~\cite{arning1996lmdd}, Local Correlation Integral (LOCI)~\cite{papadimitriou2003loci}, Lightweight Online Detector of Anomalies (LODA)~\cite{pevny2016loda}, Local Outlier Factor (LOF)~\cite{breunig2000lof}, Minimum Covariance Determinant (MCD)~\cite{rousseeuw1999mcd,hardin2004mcd}, One-class Support Vector Machine (OCSVM)~\cite{sholkopf2001ocsvm}, Subspace Outlier Detection (SOD)~\cite{kriegel2009sod}, and Stochastic Outlier Selection (SOS)~\cite{janssens2012sos}.


\subsection{Other Approaches}
\label{subsec:introduction:oyher-appraoches}

TODO


\subsection{CHAODA}
\label{subsec:introduction:chaoda}

In this paper we introduce a novel technique, Clustered Learning of Approximate Manifolds (CLAM).
This approach uses divisive hierarchical clustering to learn a manifold in a Banach space~\cite{banach1929fonctionnelles} defined by a distance metric.
In actuality, we do not require a metric.
The space may be defined by a distance \textit{function} that does not obey the triangle inequality, though this is not always optimal.
Given a learned approximate manifold, we can almost trivially implement several anomaly-detection algorithms.
In this manuscript, we present a collection of five such algorithms implemented on CLAM: CHAODA (Clustered Hierarchical Anomaly and Outlier Detection Algorithms).

The manifold learning component is derived from prior work, CHESS~\cite{ishaq2019entropy}, to accelerate approximate search on large high-dimensional datasets.
CLAM begins by divisively clustering the data until each cluster contains only one datum.
CLAM then delineates \textit{layers} of clusters at each depth in the tree.
Each layer comprises all clusters that would have been leaf nodes if the tree building had been halted at the given depth.

CLAM then builds a graph for each layer in the tree by creating edges between clusters that have overlapping volumes.
This process effectively learns the manifold on which the data lie at various resolutions, given by the depth of the layer.
This is analogous to a ``filtration'' in computational topology~\cite{carlsson2009topology}.
Once we have learned a manifold, we can ask about the cardinality of various clusters at different depths, how connected a given cluster is, or even how often a cluster is visited by random walks on the manifold.

We test our methods on 24 real-world datasets.
The datasets span a wide variety of domains, each having a different quantity of anomalous data.
We consider several different definitions of outliers and anomalies: \textbf{distance-based}, examining several classical distance-based definitions of outliers, relying on CLAM's use of distance to cluster data; \textbf{density-based}, examining the cardinality of clusters, under the hypothesis that clusters with lower cardinality are more likely to contain outliers; \textbf{graph-based}, examining several graph-theoretic methods for anomaly detection, given graphs constructed from layers of clusters.

Historically, clustering approaches have suffered from several problems.
The most common deficiencies are: the effective treatment of high dimensionality, the ability to interpret results, and the ability to scale to exponentially-growing datasets ~\cite{agrawal1998automatic}.
CLAM largely resolves these problems.

Na\"ively, distance-based approaches either require linear time in the cardinality of the data set, or quadratic time to build an index. CLAM builds an index in expected log-linear time, and all CHAODA methods are sublinear in the cardinality of the dataset.


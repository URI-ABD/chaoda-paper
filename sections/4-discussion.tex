\section{Discussion}
\label{sec:discussion}

% TODO mention reviewer re: acknowledgements in camera ready
We have presented CHAODA, an ensemble of six algorithms that use the map of the underlying manifold produced by CLAM\@.
The six individual algorithms are simple to implement on top of CLAM and, when combined into an ensemble, often outperform state-of-the-art methods.
In future work, new ideas for additional algorithms that contribute novel inductive biases can be easily incorporated into CHAODA.

While the meta-ml models in CHAODA optimize for the best ROC-AUC scores for anomaly detection, this approach can be used to optimize for any measure on any type of task.
Future work should explore tasks other than anomaly detection and come up with mathematical measures for performance on those tasks.
Variants of CHAODA could then be trained for such tasks.

CLAM uses the geometric and topological properties such as fractal dimension of the data to build a map of the low-dimensional manifold that the data occupy.
CLAM extends CHESS~\cite{ishaq2019clustered} by:
a better selection of ``poles'' (i.e. a pair of well-separated points) for partitioning clusters,
memoizing important cluster properties, and
introducing a novel graph-induction approach using a notion of optimal depths, learned via a form of ``meta-machine-learning'' and transfer learning.
While the standard approach in machine learning is to split a single dataset into many folds, CLAM and CHAODA are unusual in that they learn geometric and topological properties for inducing graphs from a \emph{set} of training datasets, and \emph{transfer} that knowledge to entirely different datasets.
Whereas CHESS was developed specifically for accelerating search,
CHAODA uses this manifold-mapping framework to discover properties of the manifold that are useful for anomaly detection.
Intuitively, we expect CHAODA to perform particularly well when the data lie on an ``interesting'' manifold, and to perform merely average when the data derive from an easily-described distribution (or ``boring'' manifold).
Just as CHESS demonstrated an acceleration of search when the data exhibited \emph{low fractal dimension} and \emph{low metric entropy}, with CHAODA, we see that AUC scores are vastly improved when the data exhibit these properties and often competitive when the data do not exhibit these properties.
CLAM is free of hyper-parameters other than the fairly standard choice of $\alpha$ in Section~\ref{subsec:methods:training-meta-ml-models}; the weights learned from the meta-ML step could vary, but we learned them once on a distinct training set of datasets.

We should note the Vertebral Column (Vert.) dataset.
Most tested algorithms perform similarly to random guessing, while CHAODA performs much worse.
We suspect this is due to how this specific dataset was collected.
Each instance represents six biomechanical attributes derived from scans of a patient's pelvis and lumbar spine.
This dataset contains 210 instances of the Abnormal class treated as inliers and 30 instances of the Normal class treated as outliers.
Each attribute has a narrow range to be in the Normal class but can have a wider range in the Abnormal class.
This causes the Normal instances to group together, while Abnormal instances remain distant from each other.
As CHAODA relies on clusters as the substrate, it assigns low scores to instances in the Normal class, i.e.\ the outliers, and high scores to those in the Abnormal class, i.e.\ the inliers.
Put plainly, CHAODA sees the manifold as the Normal class, which the ground-truth labels as outliers.

The choice of distance function could significantly impact anomaly-detection performance.
In this case, domain knowledge is likely the best way to determine the distance function of choice.
Future work should explore a more diverse collection of domain-appropriate distance functions, such as Wasserstein distance on images, Levenshtein distance on strings, and Jaccard distance on the maximal common subgraph of molecular structures.
Currently, CLAM only functions on a Banach space defined by a distance metric.
This poses a limitation on datasets that have heterogenous features, such as a mix of continuous and categorical variables.
Future work should explore linear combinations of normalized distance functions to overcome this limitation.

In this paper, we have used CHAODA (and the methods under comparison) to score entire datasets with known anomaly labels for purposes of evaluating CHAODA's accuracy.
In real-world usage, one might wish to assign anomaly scores to incoming, streaming data.
This is a simple extension: given some corpus of data (some of which may or may not be anomalous), build a CLAM tree and the induced graphs, and assign anomaly scores from the CHAODA algorithms as we have demonstrated.
Then, as each new datum arrives, simply fit it into the CLAM tree ($\mathcal{O}(\lg |V|)$ time using tree-search from CHESS) into a cluster that is found in a graph and assign it the anomaly score for that cluster.
If an incoming datum is too far from any cluster (further than any existing datum at that depth from its cluster center) then it can initialize a new cluster, which would be assigned a high anomaly score.
Thus, in general, CHAODA requires $\mathcal{O}(\lg |V|)$ time to assign an anomaly score to a new datum.

% TODO new whole paragraph discussing suitability to large, HD datasets.

CHAODA is demonstrably highly effective on large high-dimensional datasets, and so may be applied to neural networks.
Using CLAM to map a dataset where each datum represents the activation-pattern of a neural network from an input to the neural network, we would expect to detect malicious inputs to neural networks based on the intuition that malicious inputs produce atypical activation patterns.

In conclusion, we have demonstrated that by mapping the manifolds occupied by data, CLAM reveals geometric and topological structure that allows CHAODA to outperform other state-of-the-art approaches to anomaly detection, representing an actualization of the manifold hypothesis.

The source code for CLAM and CHAODA are available under an MIT license at https://github.com/URI-ABD/clam.

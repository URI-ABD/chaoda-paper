\documentclass{article}
\pdfoutput=1

\usepackage{arxiv} % uncomment for preprint

\usepackage[utf8]{inputenc} % allow utf-8 input
\usepackage[T1]{fontenc}    % use 8-bit T1 fonts
% \usepackage{hyperref}       % hyperlinks
\usepackage{url}            % simple URL typesetting
\usepackage{booktabs}       % professional-quality tables
\usepackage{amsfonts}       % blackboard math symbols
\usepackage{nicefrac}       % compact symbols for 1/2, etc.
\usepackage{microtype}      % microtypography
\usepackage{lipsum}
\usepackage{graphicx}
\usepackage{subfigure}
\usepackage{amsmath}

\DeclareMathOperator*{\argmax}{arg\,max}
\DeclareMathOperator*{\argmin}{arg\,min}

\usepackage{placeins}
% \usepackage{algpseudocode}
\usepackage{algorithm}
\usepackage{algorithmic}

% \algnewcommand\algorithmicforeach{\textbf{foreach}}
% \algdef{S}[FOR]{ForEach}[1]{\algorithmicforeach\ #1\ \algorithmicdo}


\usepackage{hyperref}

% Attempt to make hyperref and algorithmic work together better:
\newcommand{\theHalgorithm}{\arabic{algorithm}}

% Use the following line for the initial blind version submitted for review:
% \usepackage{icml2021}



\title{Clustered Hierarchical Anomaly and Outlier Detection Algorithms}



\author{
    Najib Ishaq \\
    Department of Computer Science and Statistics\\
    University of Rhode Island\\
    Kingston, RI\\
    \texttt{najib\_ishaq@uri.edu} \\
    \And
    Thomas J. Howard III \\
    Department of Computer Science and Statistics\\
    University of Rhode Island\\
    Kingston, RI\\
    \texttt{thoward27@uri.edu} \\
    \AND
    Noah M. Daniels \\
    Department of Computer Science and Statistics\\
    University of Rhode Island\\
    Kingston, RI\\
    \texttt{noah\_daniels@uri.edu} \\
}

\begin{document}
    \maketitle

% If accepted, instead use the following line for the camera-ready submission:
%\usepackage[accepted]{icml2021}
% \twocolumn[
% \icmltitle{Clustered Hierarchical Anomaly and Outlier Detection Algorithms}    

% It is OKAY to include author information, even for blind
% submissions: the style file will automatically remove it for you
% unless you've provided the [accepted] option to the icml2021
% package.

% List of affiliations: The first argument should be a (short)
% identifier you will use later to specify author affiliations
% Academic affiliations should list Department, University, City, Region, Country
% Industry affiliations should list Company, City, Region, Country

% You can specify symbols, otherwise they are numbered in order.
% Ideally, you should not use this facility. Affiliations will be numbered
% in order of appearance and this is the preferred way.
% \icmlsetsymbol{equal}{*}

% \begin{icmlauthorlist}
% \icmlauthor{Najib Ishaq}{equal,uri}
% \icmlauthor{Thomas J. Howard III}{equal,uri}
% \icmlauthor{Noah M. Daniels}{uri}
% \end{icmlauthorlist}

% \icmlaffiliation{uri}{Department of Computer Science \& Statistics, University of Rhode Island, Kingston, Rhode Island, USA}

% \icmlcorrespondingauthor{Noah M. Daniels}{noah\_daniels@uri.edu}

% % You may provide any keywords that you
% % find helpful for describing your paper; these are used to populate
% % the "keywords" metadata in the PDF but will not be shown in the document
% \icmlkeywords{Anomaly Detection, Outlier Detection, Manifold Learning}

% \vskip 0.3in
% ]

% this must go after the closing bracket ] following \twocolumn[ ...

% This command actually creates the footnote in the first column
% listing the affiliations and the copyright notice.
% The command takes one argument, which is text to display at the start of the footnote.
% The \icmlEqualContribution command is standard text for equal contribution.
% Remove it (just {}) if you do not need this facility.

%\printAffiliationsAndNotice{}  % leave blank if no need to mention equal contribution
% \printAffiliationsAndNotice{\icmlEqualContribution} % otherwise use the standard text.


    \begin{abstract}
        Anomaly and outlier detection in datasets is a long-standing problem in machine learning.
        In some cases, anomaly detection is easy, such as when data are drawn from well-characterized distributions such as the Gaussian.
        However, when data occupy high-dimensional spaces, anomaly detection becomes more difficult.
        We present CLAM (Clustered Learning of Approximate Manifolds), a fast hierarchical clustering technique that learns a manifold in a Banach space defined by a distance metric.
        CLAM induces a graph from the cluster tree, based on overlapping clusters determined by several geometric and topological features.
        On these graphs, we implement CHAODA (Clustered Hierarchical Anomaly and Outlier Detection Algorithms), exploring various properties of the graphs and their constituent clusters to compute scores of anomalousness.
        On 24 publicly available datasets, we compare the performance of CHAODA (by measure of ROC AUC) to a variety of state-of-the-art unsupervised anomaly-detection algorithms.
        Six of the datasets are used for training.
        CHAODA outperforms other approaches on 14 of the remaining 18 datasets.
    \end{abstract}

    \section{Introduction}
\label{sec:introduction}

Detecting anomalies and outliers from data is a well-studied problem in machine learning.
When data occupy easily-characterizable distributions, such as the Gaussian, the task is relatively easy: one need only identify when a datum is sufficiently far from the mean.
However, in ``big data'' scenarios, where data can occupy high-dimensional spaces, anomalous behavior becomes harder to quantify.
If the data happen to be uniformly distributed, one can conceive of simple mechanisms, such as a one-class SVM, that would be effective in any number of dimensions.
However, real-world data are rarely distributed uniformly.
Instead, data often obey the ``manifold hypothesis''~\cite{fefferman2016testing}, occupying a low-dimensional manifold in a high-dimensional embedding space.
This low-dimensional manifold may weave itself through the high-dimensional space much like a crumpled sheet of paper does in 3-dimensional space.
Detecting anomalies in such a landscape is not easy;
in particular, correctly identifying an anomalous datum that sits within the gaps of a lower-dimensional manifold presents a challenge.

Anomalies (data that do not belong to a distribution) and outliers (data which represent the extrema of a distribution) can arise from many sources:
\begin{itemize}
    \item errors in measurement or collection of data,
    \item novel, previously-unseen instances of data,
    \item normal behavior evolving into abnormal behavior
    \item and adversarial attacks as inputs to machine-learning algorithms~\cite{elsayed2018adversarial}.
\end{itemize}

Modern algorithms designed to detect anomalous behavior fail for a variety of reasons, in particular when anomalies live close to, but not on, a complex manifold in high-dimensional space.
Our approach is designed to map these complex manifolds.
Here we briefly survey contemporary approaches to anomaly detection in order to provide the context needed to understand how our approach differs.

\subsection{Clustering-based Approaches}
\label{subsec:introduction:clustering-based-approaches}

Clustering refers to techniques for grouping points in a way that provides value.
This is generally done by assigning \textit{similar} points to the same cluster.
Given a clustering and a new point, one can estimate the anomalousness of the new point by measuring its distance to its nearest cluster.

There have been few advancements in clustering techniques over the past decade~\cite{wang2019progress}.
This may be explained by the poor performance of clustering in high-dimensional space~\cite{zhang2013advancements} thus far.
The major clustering approaches are as follows.

\subsubsection{Distance-based Clustering}
\label{subsubsec:introduction:clustering-based-approaches:distance-based-clustering}
relies on some distance measure to partition data into some number of clusters.
Within this approach, the numbers and/or sizes of clusters are often predetermined: either user-specified, or chosen at random~\cite{wang2019progress}.
Some examples of distance-based clustering are
K-Means~\cite{macqueen1967some},
PAM~\cite{kaufman2009finding},
CLARANS~\cite{ng1994efficient} and
CLARA~\cite{kaufman2009finding}.

\subsubsection{Hierarchical Clustering}
\label{subsubsec:introduction:clustering-based-approaches:hierarchical-clustering}
methods utilize a tree-like structure, where points are allocated into leaf nodes~\cite{wang2019progress}.
These tree-like structures can be created bottom-up (agglomerative clustering) or top-down (divisive clustering)~\cite{agrawal1998automatic}.
A major drawback of these methods is the high cost of pairwise distance computations required to build each level of the tree.
Examples of hierarchical clustering include
MST~\cite{charles_zahn_graph_1971},
CURE~\cite{guha1998cure} and
CHAMELEON~\cite{karypis1999hierarchical}.

\subsubsection{Grid-based Clustering}
\label{subsubsec:introduction:clustering-based-approaches:grid-based-clustering}
works via segmenting the entire space into a discrete number of cells, and then scanning those cells to find regions of high density.
Utilizing a grid structure for clustering means that these algorithms typically scale well to larger datasets.
Some examples of grid-based clustering include
STING~\cite{wang1997sting},
Wavecluster~\cite{sheikholeslami2000wavecluster}, and
CLIQUE~\cite{agrawal1998automatic}.

In this paper, we compare against the following Clustering-based approaches:
\begin{itemize}
    \item Clustering Based Local Outlier Factor (CBLOF)~\cite{he2003cblof},
    \item Local Correlation Integral (LOCI)~\cite{papadimitriou2003loci},
\end{itemize}


\subsubsection{Density-based Approaches}
\label{subsec:introduction:density-based-approaches}
methods rely on finding regions of high point-density separated by regions of low point-density.
These algorithms generally do not work well when data are sparse or uniformly distributed.
Some examples of density-based clustering algorithms are
DBSCAN~\cite{ester1996density} and
DENCLUE~\cite{hinneburg1998efficient}.

In this paper, we compare against the following Density-based approaches:
\begin{itemize}
    \item Local Outlier Factor (LOF)~\cite{breunig2000lof},
\end{itemize}


\subsection{Graph-based Approaches}
\label{subsec:introduction:graph-based-approaches}

Graph-based methods build graph representations of the data by treating points, or collections of points, as nodes in a graph and drawing edges between them based of various rules.

In this paper, we compare against the following Graph-based approaches:
\begin{itemize}
    \item Connectivity-based Outlier Factor (COF)~\cite{tang2002cof},
\end{itemize}


\subsection{Distanced-based Approaches}
\label{subsec:related-works:distanced-based-approaches}

Distance-based methods find anomalous points via distance comparisons.
These methods largely employ k-Nearest Neighbors as their substrate~\cite{wang2019progress}.
Distance-based approaches tend to use the following intuitions:
\begin{itemize}
    \item points with fewer than $p$ other points within some distance $d$ are outliers;
    \item the $n$ points with the greatest distances to their $k^{th}-$nearest neighbor are outliers; or
    \item the $n$ points with the greatest average distance to their $k$ nearest neighbors are outliers.
\end{itemize}

In this paper, we compare against the following Distance-based approaches:
\begin{itemize}
    \item k-Nearest Neighbors (kNN)~\cite{ramaswamy2000efficient, sridhar2000knn, fabrizio2002knn},
    \item Subspace Outlier Detection (SOD)~\cite{kriegel2009sod},
\end{itemize}


\subsection{Other Approaches}
\label{subsec:introduction:other-appraoches}

There are several approaches to anomaly detection that do not fall neatly into any of the aforementioned categories.
These methds can often rely on Support-Vector Machines, Random Forests, or Histograms to detect outliers.

In this paper, we compare against the following miscelaneous approaches:
\begin{itemize}
    \item Histogram-Based Outlier Detection (HBOS)\cite{goldstein2012hbos},
    \item Isolation-Forest Outlier Detector (IFOREST)~\cite{tony2008iforest,tony2012iforest},
    \item One-class Support Vector Machine (OCSVM)~\cite{sholkopf2001ocsvm},
    \item Linear Model Deviation-base outlier Detection(LMDD)~\cite{arning1996lmdd},
    \item Lightweight Online Detector of Anomalies (LODA)~\cite{pevny2016loda},
    \item Minimum Covariance Determinant (MCD)~\cite{rousseeuw1999mcd,hardin2004mcd},
    \item Subspace Outlier Detection (SOD)~\cite{kriegel2009sod},
\end{itemize}


\subsection{CHAODA}
\label{subsec:introduction:chaoda}

TODO: CHAODA is an ensemble approach

In this paper we introduce a novel technique, Clustered Learning of Approximate Manifolds (CLAM).
This approach uses divisive hierarchical clustering to learn a manifold in a Banach space~\cite{banach1929fonctionnelles} defined by a distance metric.
In actuality, we do not require a metric.
The space may be defined by a distance \textit{function} that does not obey the triangle inequality, though this is not always optimal.
Given a learned approximate manifold, we can almost trivially implement several anomaly-detection algorithms.
In this manuscript, we present a collection of five such algorithms implemented on CLAM: CHAODA (Clustered Hierarchical Anomaly and Outlier Detection Algorithms).

The manifold learning component is derived from prior work, CHESS~\cite{ishaq2019entropy}, to accelerate approximate search on large high-dimensional datasets.
CLAM begins by divisively clustering the data until each cluster contains only one datum.
CLAM then delineates \textit{layers} of clusters at each depth in the tree.
Each layer comprises all clusters that would have been leaf nodes if the tree building had been halted at the given depth.

CLAM then builds a graph for each layer in the tree by creating edges between clusters that have overlapping volumes.
This process effectively learns the manifold on which the data lie at various resolutions, given by the depth of the layer.
This is analogous to a ``filtration'' in computational topology~\cite{carlsson2009topology}.
Once we have learned a manifold, we can ask about the cardinality of various clusters at different depths, how connected a given cluster is, or even how often a cluster is visited by random walks on the manifold.

We test our methods on 24 real-world datasets.
The datasets span a wide variety of domains, each having a different quantity of anomalous data.
We consider several different definitions of outliers and anomalies: \textbf{distance-based}, examining several classical distance-based definitions of outliers, relying on CLAM's use of distance to cluster data; \textbf{density-based}, examining the cardinality of clusters, under the hypothesis that clusters with lower cardinality are more likely to contain outliers; \textbf{graph-based}, examining several graph-theoretic methods for anomaly detection, given graphs constructed from layers of clusters.

Historically, clustering approaches have suffered from several problems.
The most common deficiencies are: the effective treatment of high dimensionality, the ability to interpret results, and the ability to scale to exponentially-growing datasets ~\cite{agrawal1998automatic}.
CLAM largely resolves these problems.

Na\"ively, distance-based approaches either require linear time in the cardinality of the data set, or quadratic time to build an index. CLAM builds an index in expected log-linear time, and all CHAODA methods are sublinear in the cardinality of the dataset.


    \section{Methods}
\label{sec:methods}

\subsection{CLAM}
\label{subsec:methods:clam}

We present a manifold-mapping algorithm called CLAM (Clustered Learning of Approximate Manifolds).
As input, we need a dataset and a distance function on the points in that dataset.
A dataset is a collection of $n$-points in a $D$-dimensional embedding space, $\textbf{X} = \{x_1 \dots x_n\}, x_i \in \mathbb{R}^D$.
A Distance Function takes two points in the dataset and deterministically produces a non-negative real number, $f : (\mathbb{R}^D, \mathbb{R}^D) \mapsto \mathbb{R}^+$.
The distance must be such that $f(x, x) = 0$ and $f(x, y) = f(y, x)$ $\forall x, y \in X$.
The distance function may or may not obey the triangle-inequality.
In this paper we used $L1$ and $L2$ norms, which are metrics.

\subsubsection{Clustering}
\label{subsubsec:methods:clam:clustering}

% describe creation of Cluster Tree
We start by building a divisive-hierarchical clustering of the data based on a random sampling of $\sqrt k$ points from the  $k$ points in a given cluster from the tree.
This achieves clustering in expected $\mathcal{O}(n \lg n)$ time.
This gives us a tree of clusters, with the root containing every point in the dataset, and each leaf containing a single point from the dataset.
The CLAM clustering algorithm is defined in Algorithm~\ref{alg:clam}.
% The procedure is detailed in~\cite{ishaq2019clustered}.
% We can't cite our work as "us" during the review process

\begin{algorithm} % enter the algorithm environment
\caption{Cluster} % give the algorithm a caption
\label{alg:clam} % and a label for \ref{} commands later in the document
\begin{algorithmic}[1] % enter the algorithmic environment
\STATE $minsize \leftarrow 1$
\STATE $data$, a dataset.
\STATE $d \gets 0$
\STATE $clusters = \{\}$
\STATE $n = |data|$
\WHILE{true}
    \STATE $seeds \leftarrow \sqrt{n}$ random points from $data$
    \STATE $l, r \gets \{l, r | l,r \in seeds \land l,r = \argmax d(x,y) | x,y \in seeds\}$
    \STATE $clusters[l] \gets \{x | x \in data \land d(l,x) \le d(r,x)\}$
    \STATE $clusters[r] \gets \{x | x \in data \land d(r,x) < d(l,x)\}$
    \IF{$|clusters[l]| > minsize$}
        \STATE Cluster(clusters[l])
    \ENDIF
    \IF{$|clusters[r]| > minsize$}
        \STATE Cluster(clusters[r])
    \ENDIF
\ENDWHILE
\end{algorithmic}
\end{algorithm}

These clusters have several interesting and important properties for us to consider.
These include the \textit{cardinality}, the number of points in a cluster;
\textit{center}, the geometric median of points contained in a cluster;
\textit{radius}, the distance to the farthest point from the center;
and \textit{local fractal dimension}, as described in~\cite{ishaq2019clustered}.
We can also consider individual \textit{parent-child ratios} of cardinality, radius, and local fractal dimension, as well as \textit{exponential moving averages} of those parent-child ratios along a branch of the tree.
Each of these properties is computed during the process of clustering, and stored as metadata for each cluster.
In particular, we use the parent-child ratios and the exponential moving averages of those ratios to generalize our method from a small set of datasets to a large, distinct set of datasets.

\subsubsection{Graphs}
Clusters that are close in the embedding space sometimes have overlapping volumes, i.e.\ the distance between their centers is less than or equal to the sum of their radii.
We define a graph $G=(V,E)$ with the selected clusters in one-to-one correspondence to vertices, and with an edge between two vertices if and only if their corresponding clusters overlap.
Note that any given datapoint will be a member of only one cluster at a given depth in the tree, even though it might exist inside another clusters' radii.
Thus, the clusters are not necessarily hyperspheres, but rather polytopes akin to a high-dimensional version of a Voronoi diagram~\cite{voronoi1908nouvelles}.
For our purposes, a graph also exhibits a particular invariant.
The clusters corresponding to vertices in the graph collectively contain every point in the dataset.
In addition, each point in the dataset is in exactly one cluster corresponding to exactly one vertex in the graph.

A corollary to this invariant is that the graph will never contain two vertices such that one vertex corresponds to a parent or child cluster of another vertex.
A Graph can be built from clusters at a fixed depth in the cluster-tree (a layer-graph), or from clusters from multiple different depths in the tree (an optimal-graph).
In this work we consider the cardinality of a graph to be \textit{vertex cardinality}, i.e.\ the number of vertices (clusters) in the graph.

\subsubsection{The Manifold}
% describe Manifold as containing the tree and all the graphs.
According to the ``manifold hypothesis''~\cite{fefferman2016testing},
datasets from constrained generating processes that are embedded in a high-dimensional space typically only occupy a low-dimensional manifold in that space.

The graphs discussed so far map this low-dimensional manifold in the original embedding space.
Different graphs do this at different levels of local or global resolution.
Our aim is to properly build such a graph, where different resolutions may be necessary for different regions of the manifold.
We can then apply several anomaly detection algorithms to these graphs.
These algorithms will often also incorporate information from the tree, such as parent-child relationships.

We describe some algorithms in Section~\ref{subsec:methods:individual-algorithms}.
While these algorithms are themselves fairly simple, the real challenge is in selecting the right clusters for the graphs for the algorithms to operate on.
We will demonstrate CLAM's manifold mapping to be effective enough that even these simple algorithms, more often than not, outperform state-of-the-art anomaly detection algorithms.


\subsection{Induced Graphs}
\label{subsec:methods:induced-graphs}

The heart of the problem with our methods for anomaly detection is building the right graph to represent the underlying manifold.
One could try using every possible combination of clusters to form every possible graph but this leads to combinatorial explosion.
Instead, we must intelligently select those clusters that, when used to build the graph, perform best for anomaly detection.

Area Under the Curve (AUC) of the Receiver Operating Characteristic (ROC) is often used to measure the performance of anomaly detectors;
we wish to choose clusters that are expected to maximize this measure.
We do this by learning a function that takes a cluster and predicts its contribution to AUC if that cluster were selected for the graph.
Any function $f: Cluster \mapsto \mathbb{R}^+$ will suffice.
Importantly, we will learn this function on datasets \emph{completely unrelated to} the ones on which we test; we are learning not dataset-specific parameters, but geometric and topological properties of data.

We chose a linear regression model and a regression tree model to fill this role.
Such models need some data to train with.
To generate this data, we took a random sample from the datasets described in Section~\ref{subsec:methods:datasets}, choosing six datasets at random which had between 1,000 and 100,000 samples.
We generate CLAM manifolds for these training datasets, use the linear regression  and regression tree models to learn from these datasets, and apply the results to an entirely different collection of datasets.
The process is as follows.

We generate the initial training data by considering layer-graphs.
For each such graph, we calculate the means of the parent-child cardinality ratio, radius ratio, and local fractal dimension ratio;
along with the exponential moving averages of these ratios, as described in Section~\ref{subsubsec:methods:clam:clustering}.
These ratios form the feature vector for that one graph.
For each individual algorithm described in Section~\ref{subsec:methods:individual-algorithms}, we apply it to the graph and obtain an AUC ROC\@.
We then train the linear regression and decision tree models to predict this AUC from the feature vector extracted from the graph.
The result is a pair of models for each indivisual algorithm of CHAODA, trained from the six training datasets;
the weights learned here are then used for all test datasets, regardless of their dimensionality or other properties.
Figure~\ref{fig:methods:graph-generation} illustrates how a graph can be induced by clusters and their overlaps at different depths of a tree.

\begin{figure}[ht!]
    \centering
    \includegraphics[width=3in]{images/tree-graph.pdf}
    \caption{Cartoon illustration of how CLAM induces a graph from a cluster tree.
        Dots in the tree represent cluster centers;
        blue dots represent cluster centers chosen as graph vertices.
        Circles around those centers represent the volume of a cluster (the radius is the distance from the center to the furthest point contained within that cluster).
        Gray arrows point to the induced subgraphs, which are indicated in blue below the horizontal line.}
    \label{fig:methods:graph-generation}
\end{figure}


% describe individual algorithms.
\subsection{Individual Algorithms}
\label{subsec:methods:individual-algorithms}

Here we describe several simple methods for anomaly detection, each of which uses a graph of clusters from CLAM to calculate an anomalousness score for each datapoint.
For each algorithm, $scores$ is a dictionary of clusters and their outlier scores,
$V$ is the set of clusters in a graph,
$E$ is the set of edges in a graph, and
the cardinality of a cluster $c$ is denoted by $|c|$, the number of points in that cluster.

\subsubsection{Relative Cluster Cardinality}
We measure the anomalousness of a point by the cardinality of the cluster that the point belongs to relative to the cardinalities of the other clusters in the graph.
Points in the same cluster are considered equally anomalous and points in clusters with lower cardinalities are considered more anomalous than points in clusters with higher cardinalities. The algorithm is defined in Algorithm~\ref{alg:rclc}. The time complexity of this algorithm is $O(|V|)$.

% \begin{itemize}
%     \item for each cluster $c$ in graph $g$:
%     \begin{itemize}
%         \item $scores[c] \leftarrow -|c|$.
%     \end{itemize}
% \end{itemize}

\begin{algorithm}[h]
    \caption{Relative Cluster Cardinality}
    \label{alg:rclc}
\begin{algorithmic}[1]
    \REQUIRE $G$, a graph
    \FOR {cluster $c \in G$}
    \STATE $scores[c] \gets -|c|$
    \ENDFOR
\end{algorithmic}
\end{algorithm}

\subsubsection{Relative Component Cardinality}
We use the usual definition of connected components: no two nodes from different components have an edge between them, and every pair of nodes in the same component has a path connecting them.
Consider the relative cardinalities of each component in much the same way as we considered the relative cardinalities of clusters in the Relative Cluster Cardinality method.
Points in clusters (nodes) in smaller components are considered more anomalous than points in clusters in larger components,
and points in clusters in the same component are considered equally anomalous.
The algorithm is defined in Algorithm~\ref{alg:rcc}.
This algorithm first finds the components of the graph, so its time complexity is $O(|E| + |V|)$.

\begin{algorithm}[h]
    \caption{Relative Component Cardinality}
    \label{alg:rcc}
\begin{algorithmic}[1]
    \REQUIRE $G$, a graph
    \FOR {component $C \in G$}
        \STATE $scores[C] \gets -|C|$
    \ENDFOR
\end{algorithmic}
\end{algorithm}

\subsubsection{Graph Neighborhood Size:}
Given the graph with clusters and edges, consider the number of clusters reachable from a starting cluster within a given graph distance $k$.
We call this number the \textit{graph-neighborhood} of the starting cluster.
With $k$ small compared to the diameter of the graph, we consider the relative sizes of \textit{graph-neighborhood} of the clusters.
Points in clusters with small graph-neighborhoods are considered more anomalous than points in clusters with large graph-neighborhoods.
The algorithm is defined in Algorithm~\ref{alg:gns}.
This algorithm is dominated by computing the eccentricity of each cluster, so its time complexity is $O(|E| \cdot |V|)$.

\begin{algorithm}[h]
    \caption{Graph Neighborhood Size}
    \label{alg:gns}
\begin{algorithmic}[1]
    \REQUIRE $G$, a graph
    \REQUIRE $f \in \mathbb{R}$ in the range $(0,1]$ ($0.25$ by default).
    \FOR {cluster $c \in G$}
        \STATE $e_c \gets$ the eccentricity of $c$
        \STATE $s \gets e_c \cdot f$
        \STATE perform a breadth-first traversal from $c$ with $s$ steps
        \STATE $v \gets$ the number of unique clusters visited by the traversal
        \STATE $scores[c] \gets -v$
    \ENDFOR
\end{algorithmic}
\end{algorithm}

\subsubsection{Child-Parent Cardinality Ratio}
% As described in CHESS~\cite{ishaq2019clustered},
As described in Section~\ref{subsubsec:methods:clam:clustering}, a cluster is partitioned by using its two maximally distant points as poles.
All points in a cluster are split among the children by whichever pole they are closer to.
Consider the fraction of points in a cluster that are assigned to each child.
If a child cluster contains only a small fraction of its parent's points, then we consider the points in that child cluster to be more anomalous.
These child-parent cardinality ratios are accumulated for each point down its branch in the tree, terminating when the child cluster is a node in the induced optimal graph.
Points with a low value of these accumulated ratios are considered more anomalous than points with a higher value.
The algorithm is defined in Algorithm~\ref{alg:cpcr}.
The time complexity of calculating the ratios is $O(|V|)$at clustering time, which is amortized over scoring the points; the time complexity of scoring a point is $O(1)$.

\begin{algorithm}[h]
    \caption{Child-Parent Cardinality Ratio}
    \label{alg:cpcr}
\begin{algorithmic}[1]
    \REQUIRE $G$, a graph
    \FOR {cluster $c \in G$}
        \STATE $p \gets$ the parent cluster of $c$
        \STATE $scores[c] \gets \frac{|p|}{|c|}$
    \ENDFOR
\end{algorithmic}
\end{algorithm}

\subsubsection{Random Walks}
We perform long random walks on each component of a graph, and we count the number of times that each cluster was visited.
We consider the relative visitation counts among the clusters to be inversely related to their anomalousness.
The least visited clusters are the most anomalous, and the most visited clusters are the least anomalous.
The algorithm is defined in Algorithm~\ref{alg:rw}.
This algorithm is dominated by the computation of the transition matrix for the graph.
Its worst-case time complexity is $O(|V|^2)$, but is often better because the transition matrix is computed separately for each component in the graph.

\begin{algorithm}[h]
    \caption{Random Walks}
    \label{alg:rw}
\begin{algorithmic}[1]
    \REQUIRE $G = (V,E)$, a graph
    \STATE Initialize $v[c] \gets 0 \ \forall c \in G$
    \FOR {cluster $c \in G$}
        \STATE $c' \gets c$
        \FOR{$i=1$ {\bfseries to} $10 \cdot |V|$}
            \STATE $c' \gets n(d),$ a random neighbor of $d$
            \STATE increment $v[c']$
        \ENDFOR
    \ENDFOR
    \FOR {cluster $c \in G$}
        \STATE $scores[c] \gets \frac{|p|}{|c|}$
    \ENDFOR
\end{algorithmic}
\end{algorithm}

\subsubsection{Stationary Probabilities}
We compute the transition probability matrix of each component of a graph that contains at least two clusters.
We then compute successive squares of this matrix.
This process will eventually converge as long as the graph is connected and aperiodic~\cite{levin2017markov}, and we find this convergent matrix.
The sum of the values along a row is inversely related to the anomalousness of the respective cluster.
The algorithm is defined in Algorithm~\ref{alg:sp}.
This algorithm is dominated by the computation of the transition matrix for each component in the graph.
Its worst-case time complexity is $O(|V|^2)$, but is often better in practice.

\begin{algorithm}[h]
    \caption{Stationary Probabilities}
    \label{alg:sp}
\begin{algorithmic}[1]
    \REQUIRE $G$, a graph
    \FOR {component $C \in G$}
        \STATE $M \gets$ the transition matrix for $C$
        \REPEAT
            \STATE $M \gets M^2$
        \UNTIL $M$ converges
        \FOR {cluster $c \in C$}
            \STATE $s \gets $ the row from $M$ corresponding to $c$
            \STATE $scores[c] \gets -\Sigma(s)$ 
        \ENDFOR
    \ENDFOR
\end{algorithmic}
\end{algorithm}


\subsection{Normalization}
\label{subsec:methods:normalization}

The individual methods described in~\ref{subsec:methods:individual-algorithms} produce outlier scores for each cluster, rather than for each point.
We remedy this by assigning each point the score of the cluster it belongs to.
Since our graphs guarantee that each point is in exactly one cluster, and that every point from the dataset is accounted for, this assigns an outlier score to each point in a dataset.

However, the individual methods still produce scores in a wide range of values.
The only common feature among these scores is that low scores correspond to inliers and high scores correspond to outliers.
Therefore, we cannot directly compare scores from different methods, which is needed for ensemble methods.
We normalize to the range $[0, 1]$, using gaussian normalization by default, though we include sigmoid normalization and min-max normalization as options in our implementation.
% TODO: Add equations and/or citations for normalization methods.
% if room.

\subsection{Ensemble}
\label{subsec:methods:ensemble}

Given a dataset and a list of distance functions for the dataset, we start by building a CLAM tree for each distance function on the dataset.
we extract the parent-child ratios and the exponential moving averages of those ratios for each cluster in each tree.
Given the set of models learned from the training datasets described in Section~\ref{subsec:methods:induced-graphs}, we use each model to rank every cluster, normalized by cardinality.
The highest ranked clusters from each tree are then used to build a graph.
This gives us one graph for each combination of individual algorithm, meta-ML model, and distance function used.
We used the six individual algorithms described in~\ref{subsec:methods:individual-algorithms}, linear regression and regression trees as the meta-ML models, and L1-norm and L2-norm as distance functions;
thus, we have upto $24$ graphs for each dataset.

Each graph is then used with its corresponding individual algorithm to calculate an outlier score for each point in the dataset.
These outlier scores are then combined into an ensemble by simply taking the mean of all the scores for each point.
We present the AUC scores from this ensemble in Section~\ref{sec:results}.

\subsubsection{A Note on Runtime Performance}

During testing, we noticed that even though we often see $|V| \ll n $, where $n$ is the number of points in the dataset and $V$ is the set of vertices in a graph, the more expensive methods from~\ref{subsec:methods:individual-algorithms} took too long to run.
We remedy this by only running the expensive algorithms, i.e.\ Graph Neighborhood Size, Random Walks, and Stationary Probabilities, when $|V| < max(128, \lfloor \sqrt n \rfloor)$.
We compared the effect on AUC of using this threshold and present the results in Table~\ref{table:results:test-performance} under the CHAODA-fast rows as compared to not using the threshold under the CHAODA rows.
CHAODA-fast exhibits negligible difference from CHAODA in performance, so we
set CHAODA-fast as the default in our implementation.


\subsection{Datasets}
\label{subsec:methods:datasets}

We sourced 24 datasets, all from Outlier Detection Datasets (ODDS)~\cite{rayana2016odds}, for training CHAODA and testing its performance.
All of these datasets were adapted from the UCI Machine Learning Repository (UCIMLR)~\cite{UCIMLR}, and were standardized, by ODDS, for benchmarks on anomaly and outlier detection.

% TODO: Did we repeat this experiment with a different six? Was it actually at random? Najib: I'm planning on upping this to 8 datasets, and then doing 4-fold cross-validation. This selection was completely random.

Of these 24 datasets, we selected a random sample of 6 datasets to use for training CHAODA\@.
The training datasets from ODDS were: annthyroid, mnist, pendigits, satellite, shuttle, and thyroid.
The remaining datasets were used to measure the performance of CHAODA\@.
Our performance on the test set of datasets is shown in Table~\ref{table:results:test-performance}.
We provide more details on the datasets, including their size, dimensionality, and fraction of outliers, in the supplementary materials.

Note that CHAODA is an unsupervised algorithm for outlier detection.
As such, we compare only against other unsupervised algorithms. %TODO but REPEN?
We compared against $18$ unsupervised algorithms implemented in pyOD~\cite{zhao2019pyod}, as well as (list others here)~\cite{}. % TODO fill this in
% looks like just loci in the table, and rs-hash for a couple datasets, is this right? if so, let's clarify here.
A supervised version of CHAODA will be possible future work.

    \section{Results}
\label{sec:results}

We have sourced 24 datasets from ODDS~\cite{rayana2016odds}.
These datasets had already been preprocessed to standardize them for outlier detection benchmarks.

% TODO: Did we repeat this experiment with a different six? Was it actually at random? Najib: I'm planning on upping this to 8 datasets, and then doing 4-fold cross-validation. This selection was completely random. np.random.seed(42)
Of these 24 datasets, we selected a random sample of 6 datasets, shown in Table~\ref{table:results:train-performance}, to use for training CHAODA\@.
The remaining datasets were used to measure the performance of CHAODA\@.
The performance on the test set of datasets is split among Tables~\ref{table:results:test-performance-1} and~\ref{table:results:test-performance-2}.
Each column shows the AUC ROC scores of CHAODA and every competitor.
The best, i.e.\ highest, score, and every score with 0.02 of that best score, are presented in bold.
% TODO: We should definitely let these all run for more than 10 minutes, no?
% Najib: Yes. Later. I'll give each method several hours.
If a method took longer than 10 hours on a dataset, we terminated that run and marked the corresponding cell with ``\textit{time}''.

Note that CHAODA is an unsupervised algorithm for outlier detection.
As such, we compare only against other unsupervised algorithms.
A supervised version of the algorithm is on its way.
Once implemented, we will add comparisons against supervised algorithms.

Also note that we implemented CLAM and CHAODA entirely in Python.
While we use the provided wrappers for each method we compare against, those methods are often implemented in C and C++, with heavy usage of header-only libraries and template-meta-programming (Noah, HAALP!).
Several other algorithms utilize deep learning frameworks like tensorflow and leverage the GPU.
Therefore, the comparison on wall-clock time performance is not truly fair to CHAODA.
We are reimplementing CLAM and CHAODA in Rust.
With the Rust implementation, we expect a two-orders-of-magnitude improvement in wall-clock times.

% TODO summarize results in text here, but I don't want to rewrite this twice, so let's finalize everything first.

% Was this repeated? Reads like we picked the six we don't do well on so that we could do well. Najib: Technically, CHAODA won 3/6.
% NMD: I'm leaving this out until we do cross-validation with more datasets in test. We didn't cherry-pick those, and need not make it look like we did.
% Interestingly, CHAODA did not perform particularly well on the training set of datasets.
% However, on the test set, CHAODA outperformed all competitors on 10 of the 18 datasets.


\begin{table*}[!t]
\renewcommand{\arraystretch}{1.25}
\caption{Performance on Train Datasets}
\label{table:results:train-performance}
\centering
\begin{tabular}{|c|c|c|c|c|c|c|}
\hline
\textbf{model} & \textbf{annthyroid} & \textbf{mnist} & \textbf{pendigits} & \textbf{satellite} & \textbf{shuttle} & \textbf{thyroid} \\
\hline
    CHAODA-fast &                0.64 &  \textbf{0.78} &      \textbf{0.94} &      \textbf{0.79} &             0.51 &    \textbf{0.89} \\
\hline
            ABOD &                0.50 &           0.60 &               0.53 &               0.51 &             0.54 &             0.50 \\
\hline
    AutoEncoder &                0.69 &           0.67 &               0.58 &               0.63 &             0.94 &             0.88 \\
\hline
            CBLOF &                0.59 &           0.62 &               0.59 &               0.68 &    \textbf{0.99} &             0.87 \\
\hline
            COF &                0.59 &           0.56 &               0.53 &               0.56 &             0.52 &             0.49 \\
\hline
            HBOS &       \textbf{0.84} &           0.53 &               0.52 &               0.62 &             0.74 &             0.86 \\
\hline
        IFOREST &                0.70 &           0.61 &               0.63 &               0.70 &             0.91 &    \textbf{0.91} \\
\hline
            KNN &                0.65 &           0.65 &               0.51 &               0.56 &             0.53 &             0.56 \\
\hline
            LMDD &                0.52 &           0.59 &               0.56 &               0.42 &             0.92 &             0.70 \\
\hline
            LOCI &         \textit{TO} &    \textit{TO} &        \textit{TO} &        \textit{TO} &      \textit{TO} &      \textit{TO} \\
\hline
            LODA &                0.63 &           0.66 &               0.57 &               0.65 &             0.96 &    \textbf{0.90} \\
\hline
            LOF &                0.60 &           0.57 &               0.52 &               0.57 &             0.53 &             0.49 \\
\hline
            MCD &                0.72 &           0.57 &               0.53 &               0.58 &             0.96 &             0.85 \\
\hline
        MOGAAL &                0.46 &    \textit{TO} &               0.67 &               0.59 &      \textit{TO} &             0.49 \\
\hline
            OCSVM &                0.62 &           0.63 &               0.59 &               0.62 &    \textbf{0.97} &             0.78 \\
\hline
            SOD &                0.64 &           0.55 &               0.52 &               0.52 &      \textit{TO} &             0.53 \\
\hline
        SOGAAL &                0.46 &           0.57 &               0.57 &               0.60 &             0.96 &             0.49 \\
\hline
            SOS &                0.50 &           0.52 &               0.52 &               0.47 &      \textit{TO} &             0.50 \\
\hline
            VAE &                0.69 &           0.67 &               0.58 &               0.63 &             0.94 &             0.88 \\
\hline
\end{tabular}
\end{table*}

% TODO: Add to supplements

% \begin{table*}[!t]
% \renewcommand{\arraystretch}{1.25}
% \caption{Time taken, in seconds, on Train Datasets}
% \label{table:results:train-time}
% \centering
% \begin{tabular}{|c|c|c|c|c|c|c|}
% \hline
% \textbf{model} & \textbf{annthyroid} & \textbf{mnist} & \textbf{pendigits} & \textbf{satellite} & \textbf{shuttle} & \textbf{thyroid} \\
% \hline
%     CHAODA-fast &              161.85 &        6221.64 &             617.93 &             425.03 &             5.48 &           117.97 \\
% \hline
%             ABOD &                3.42 &          16.39 &               2.83 &               3.53 &            21.58 &             1.25 \\
% \hline
%     AutoEncoder &               22.80 &          40.54 &              23.87 &              26.74 &           158.72 &            12.84 \\
% \hline
%             CBLOF &                1.01 &           1.14 &               0.23 &               0.30 &             0.71 &             0.21 \\
% \hline
%             COF &              215.63 &         277.91 &             199.03 &             176.35 &         12106.63 &            54.82 \\
% \hline
%             HBOS &                1.58 &  \textbf{0.06} &      \textbf{0.01} &      \textbf{0.02} &    \textbf{0.03} &    \textbf{0.00} \\
% \hline
%         IFOREST &                0.73 &           1.41 &               0.80 &               0.84 &             3.24 &             0.54 \\
% \hline
%             KNN &                0.84 &          14.30 &               1.25 &               2.05 &             9.13 &             0.42 \\
% \hline
%             LMDD &              107.05 &        1383.19 &             197.48 &             371.63 &          6274.43 &            32.69 \\
% \hline
%             LOCI &         \textit{TO} &    \textit{TO} &        \textit{TO} &        \textit{TO} &      \textit{TO} &      \textit{TO} \\
% \hline
%             LODA &                0.13 &           0.13 &               0.12 &               0.11 &             0.59 &    \textbf{0.08} \\
% \hline
%             LOF &                0.26 &          14.79 &               0.90 &               1.71 &             6.30 &    \textbf{0.09} \\
% \hline
%             MCD &                1.50 &          20.56 &               3.01 &              10.39 &            10.78 &             1.03 \\
% \hline
%         MOGAAL &              594.69 &    \textit{TO} &             561.40 &             513.84 &      \textit{TO} &           280.80 \\
% \hline
%             OCSVM &                2.70 &          11.30 &               3.04 &               3.83 &           162.18 &             0.68 \\
% \hline
%             SOD &              262.80 &         281.86 &             195.81 &             253.71 &      \textit{TO} &            91.48 \\
% \hline
%         SOGAAL &               61.14 &          68.03 &              55.40 &              50.07 &           460.66 &            31.43 \\
% \hline
%             SOS &              101.82 &          54.22 &              46.02 &              42.62 &      \textit{TO} &            42.10 \\
% \hline
%             VAE &               27.95 &          50.16 &              29.63 &              33.51 &           182.23 &            16.18 \\
% \hline
% \end{tabular}
% \end{table*}


\begin{table*}[!t]
\renewcommand{\arraystretch}{1.25}
\caption{Performance on the first half of the Test Datasets}
\label{table:results:test-performance-1}
\centering
\begin{tabular}{|c|c|c|c|c|c|c|c|c|c|}
\hline
\textbf{model} & \textbf{arrhythmia} & \textbf{breastw} & \textbf{cardio} & \textbf{cover} & \textbf{glass} & \textbf{http} & \textbf{ionosphere} & \textbf{lympho} & \textbf{mammo} \\
\hline
    CHAODA-fast &       \textbf{0.77} &    \textbf{0.97} &   \textbf{0.81} &  \textbf{0.71} &  \textbf{0.70} & \textbf{1.00} &                0.88 &   \textbf{0.98} &  \textbf{0.85} \\
\hline
            ABOD &                0.62 &             0.50 &            0.49 &           0.51 &           0.53 &          0.50 &                0.85 &            0.80 &           0.50 \\
\hline
    AutoEncoder &                0.65 &             0.91 &            0.74 &           0.52 &           0.54 &          0.51 &                0.65 &            0.83 &           0.51 \\
\hline
            CBLOF &                0.70 &             0.83 &            0.57 &    \textit{EX} &           0.54 &   \textit{EX} &                0.86 &            0.83 &           0.50 \\
\hline
            COF &                0.65 &             0.26 &            0.50 &           0.50 &           0.59 &          0.51 &                0.81 &            0.83 &           0.51 \\
\hline
            HBOS &                0.65 &             0.93 &            0.58 &           0.49 &           0.48 &          0.51 &                0.36 &            0.91 &           0.50 \\
\hline
        IFOREST &                0.72 &             0.91 &            0.69 &           0.50 &           0.54 &          0.53 &                0.77 &            0.83 &           0.59 \\
\hline
            KNN &                0.68 &             0.84 &            0.51 &           0.51 &           0.54 &          0.51 &       \textbf{0.90} &            0.83 &           0.51 \\
\hline
            LMDD &                0.68 &             0.64 &            0.60 &           0.49 &           0.54 &          0.51 &                0.67 &            0.65 &           0.56 \\
\hline
            LOCI &                0.62 &      \textit{TO} &     \textit{TO} &    \textit{TO} &           0.58 &   \textit{TO} &                0.58 &            0.90 &    \textit{TO} \\
\hline
            LODA &                0.65 &             0.93 &            0.60 &           0.52 &           0.48 &          0.51 &                0.63 &            0.48 &           0.52 \\
\hline
            LOF &                0.67 &             0.30 &            0.49 &           0.50 &           0.54 &          0.51 &                0.79 &            0.83 &           0.53 \\
\hline
            MCD &                0.65 &             0.94 &            0.55 &           0.50 &           0.48 &          0.50 &       \textbf{0.90} &            0.83 &           0.51 \\
\hline
        MOGAAL &                0.42 &             0.40 &            0.45 &    \textit{TO} &           0.59 &   \textit{TO} &                0.36 &            0.48 &    \textit{TO} \\
\hline
            OCSVM &                0.70 &             0.77 &            0.70 &           0.56 &           0.54 &          0.50 &                0.68 &            0.83 &           0.60 \\
\hline
            SOD &                0.59 &             0.77 &            0.48 &    \textit{TO} &           0.54 &   \textit{TO} &                0.84 &            0.65 &           0.51 \\
\hline
        SOGAAL &                0.48 &             0.30 &            0.45 &           0.61 &           0.59 &          0.51 &                0.36 &            0.48 &           0.50 \\
\hline
            SOS &                0.51 &             0.50 &            0.50 &    \textit{TO} &           0.48 &   \textit{TO} &                0.72 &            0.48 &    \textit{TO} \\
\hline
            VAE &                0.65 &    \textbf{0.95} &            0.74 &           0.52 &           0.48 &          0.51 &                0.65 &            0.83 &           0.56 \\
\hline
\end{tabular}    
\end{table*}

% TODO: Add to supplements

% \begin{table*}[!t]
% \renewcommand{\arraystretch}{1.25}
% \caption{Time taken, in seconds, on the first half of the Test Datasets}
% \label{table:results:test-time-1}
% \centering
% \begin{tabular}{|c|c|c|c|c|c|c|c|c|c|}
% \hline
% \textbf{model} & \textbf{arrhythmia} & \textbf{breastw} & \textbf{cardio} & \textbf{cover} & \textbf{glass} & \textbf{http} & \textbf{ionosphere} & \textbf{lympho} & \textbf{mammo} \\
% \hline
%     CHAODA-fast &                6.70 &             4.71 &          135.17 &        1485.24 &           1.06 &        119.14 &                2.64 &            1.08 &          46.42 \\
% \hline
%             ABOD &                0.34 &             0.20 &            0.72 &          24.02 &  \textbf{0.07} &         19.08 &                0.13 &   \textbf{0.05} &           3.82 \\
% \hline
%     AutoEncoder &                8.00 &             6.18 &            9.05 &         183.70 &           3.99 &        154.20 &                4.99 &            3.79 &          35.26 \\
% \hline
%             CBLOF &                0.16 &             0.13 &            0.17 &    \textit{EX} &  \textbf{0.06} &   \textit{EX} &       \textbf{0.07} &   \textbf{0.05} &           0.20 \\
% \hline
%             COF &                1.24 &             1.84 &           12.63 &       21681.15 &           0.24 &      21228.44 &                0.60 &            0.14 &         513.46 \\
% \hline
%             HBOS &       \textbf{0.08} &    \textbf{0.00} &   \textbf{0.01} &           1.13 &  \textbf{0.00} & \textbf{0.01} &       \textbf{0.01} &   \textbf{0.01} &  \textbf{0.01} \\
% \hline
%         IFOREST &                0.43 &             0.34 &            0.43 &           4.61 &           0.30 &          3.95 &                0.33 &            0.30 &           0.95 \\
% \hline
%             KNN &                0.19 &    \textbf{0.07} &            0.30 &          11.46 &  \textbf{0.02} &          7.47 &       \textbf{0.05} &   \textbf{0.02} &           1.58 \\
% \hline
%             LMDD &               14.85 &             2.62 &           22.12 &       11579.93 &           0.57 &       3873.99 &                1.73 &            0.42 &         243.01 \\
% \hline
%             LOCI &              307.04 &      \textit{TO} &     \textit{TO} &    \textit{TO} &          25.68 &   \textit{TO} &              120.92 &            9.78 &    \textit{TO} \\
% \hline
%             LODA &       \textbf{0.04} &    \textbf{0.03} &   \textbf{0.05} &           0.81 &  \textbf{0.03} &          0.66 &       \textbf{0.03} &   \textbf{0.03} &           0.17 \\
% \hline
%             LOF &                0.16 &    \textbf{0.01} &            0.18 &          10.24 &  \textbf{0.00} &          1.93 &       \textbf{0.01} &   \textbf{0.00} &           0.59 \\
% \hline
%             MCD &                5.08 &             0.64 &            0.89 &          28.65 &  \textbf{0.05} &          8.21 &                0.15 &   \textbf{0.05} &           2.11 \\
% \hline
%         MOGAAL &               46.46 &            42.09 &          116.73 &    \textit{TO} &          40.67 &   \textit{TO} &               40.86 &           37.84 &    \textit{TO} \\
% \hline
%             OCSVM &       \textbf{0.10} &    \textbf{0.02} &            0.23 &         257.95 &  \textbf{0.00} &        257.28 &       \textbf{0.01} &   \textbf{0.00} &           6.37 \\
% \hline
%             SOD &                1.19 &             1.89 &           13.39 &    \textit{TO} &           0.31 &   \textit{TO} &                0.76 &            0.20 &         521.25 \\
% \hline
%         SOGAAL &                6.70 &             5.03 &           14.43 &         591.14 &           3.95 &        597.17 &                4.71 &            3.96 &          92.27 \\
% \hline
%             SOS &                0.69 &            47.40 &            4.11 &    \textit{TO} &           0.18 &   \textit{TO} &                0.33 &            0.11 &    \textit{TO} \\
% \hline
%             VAE &               10.00 &             7.75 &           10.89 &         223.09 &           5.27 &        175.59 &                6.66 &            5.21 &          40.87 \\
% \hline
% \end{tabular}
% \end{table*}


\begin{table*}[!b]
\renewcommand{\arraystretch}{1.25}
\caption{Performance on the second half of the Test Datasets}
\label{table:results:test-performance-2}
\centering
\begin{tabular}{|c|c|c|c|c|c|c|c|c|c|}
\hline
\textbf{model} & \textbf{musk} & \textbf{optdigits} & \textbf{pima} & \textbf{satimage-2} & \textbf{smtp} & \textbf{vertebral} & \textbf{vowels} &  \textbf{wbc} & \textbf{wine} \\
\hline
    CHAODA-fast & \textbf{1.00} &      \textbf{0.96} &          0.63 &       \textbf{1.00} & \textbf{0.92} &               0.29 &            0.71 & \textbf{0.97} & \textbf{0.99} \\
\hline
            ABOD &          0.47 &               0.54 &          0.60 &                0.53 &          0.50 &               0.49 &   \textbf{0.75} &          0.50 &          0.43 \\
\hline
    AutoEncoder &          0.63 &               0.48 &          0.57 &                0.71 &          0.50 &               0.49 &            0.51 &          0.77 &          0.51 \\
\hline
            CBLOF & \textbf{1.00} &               0.52 &          0.64 &                0.90 &          0.50 &               0.49 &            0.52 &          0.82 &          0.46 \\
\hline
            COF &          0.53 &               0.52 &          0.54 &                0.56 &          0.50 &               0.51 &            0.71 &          0.47 &          0.46 \\
\hline
            HBOS & \textbf{1.00} &               0.60 &          0.55 &                0.49 &          0.68 &               0.47 &            0.56 &          0.77 &          0.57 \\
\hline
        IFOREST &          0.97 &               0.50 & \textbf{0.65} &                0.94 &          0.50 &               0.45 &            0.63 &          0.72 &          0.51 \\
\hline
            KNN &          0.51 &               0.51 &          0.60 &                0.61 &          0.53 &               0.47 &            0.72 &          0.51 &          0.47 \\
\hline
            LMDD &          0.48 &               0.49 &          0.37 &                0.49 &          0.65 &               0.43 &            0.49 &          0.80 &          0.62 \\
\hline
            LOCI &   \textit{TO} &        \textit{TO} &   \textit{TO} &         \textit{TO} &   \textit{TO} &               0.49 &     \textit{TO} &          0.72 &          0.46 \\
\hline
            LODA &          0.54 &               0.51 &          0.62 &                0.69 &          0.57 &               0.43 &            0.51 &          0.82 &          0.57 \\
\hline
            LOF &          0.50 &               0.53 &          0.55 &                0.55 &          0.50 &               0.49 &            0.69 &          0.50 &          0.46 \\
\hline
            MCD &          0.97 &               0.48 & \textbf{0.66} &                0.61 &          0.50 &               0.45 &            0.63 &          0.60 &          0.46 \\
\hline
        MOGAAL &          0.48 &               0.48 &          0.61 &                0.49 &   \textit{TO} &               0.51 &            0.48 &          0.60 &          0.46 \\
\hline
            OCSVM &          0.48 &               0.49 &          0.56 &                0.84 &          0.50 &               0.49 &            0.50 &          0.82 &          0.46 \\
\hline
            SOD &          0.51 &               0.51 &          0.56 &                0.58 &   \textit{TO} &               0.45 &            0.66 &          0.60 &          0.46 \\
\hline
        SOGAAL &          0.48 &               0.52 &          0.48 &                0.49 &          0.62 &      \textbf{0.54} &            0.48 &          0.47 &          0.46 \\
\hline
            SOS &          0.52 &               0.52 &          0.51 &                0.52 &   \textit{TO} &               0.49 &            0.59 &          0.52 &          0.46 \\
\hline
            VAE &          0.63 &               0.48 &          0.61 &                0.71 &          0.50 &               0.45 &            0.51 &          0.77 &          0.67 \\
\hline
\end{tabular}
\end{table*}

% TODO: Add to supplements

% \begin{table*}[!b]
% \renewcommand{\arraystretch}{1.25}
% \caption{Time taken, in seconds, on the second half of the Test Datasets}
% \label{table:results:test-time-2}
% \centering
% \begin{tabular}{|c|c|c|c|c|c|c|c|c|c|}
% \hline
% \textbf{model} & \textbf{musk} & \textbf{optdigits} & \textbf{pima} & \textbf{satimage-2} & \textbf{smtp} & \textbf{vertebral} & \textbf{vowels} &  \textbf{wbc} & \textbf{wine} \\
% \hline
%     CHAODA-fast &        858.15 &            3279.67 &         12.36 &              358.49 &        272.61 &               1.88 &          123.34 &          5.26 &          0.39 \\
% \hline
%             ABOD &          4.39 &               6.81 &          0.26 &                3.20 &         18.00 &      \textbf{0.08} &            0.51 &          0.14 & \textbf{0.04} \\
% \hline
%     AutoEncoder &         23.40 &              24.29 &          4.65 &               23.73 &        195.61 &               2.85 &            6.98 &          4.70 &          3.31 \\
% \hline
%             CBLOF &          0.27 &               0.44 & \textbf{0.08} &                0.31 &          0.65 &      \textbf{0.06} &   \textbf{0.10} & \textbf{0.08} & \textbf{0.05} \\
% \hline
%             COF &         45.60 &             119.57 &          2.40 &              141.84 &      21283.99 &               0.28 &            8.06 &          0.68 &          0.11 \\
% \hline
%             HBOS & \textbf{0.07} &      \textbf{0.04} & \textbf{0.00} &       \textbf{0.02} & \textbf{0.01} &      \textbf{0.00} &   \textbf{0.01} & \textbf{0.01} & \textbf{0.00} \\
% \hline
%         IFOREST &          0.93 &               0.92 &          0.35 &                0.79 &          3.90 &               0.30 &            0.40 &          0.33 &          0.29 \\
% \hline
%             KNN &          3.50 &               5.53 & \textbf{0.09} &                1.84 &          6.83 &      \textbf{0.03} &            0.18 & \textbf{0.05} & \textbf{0.01} \\
% \hline
%             LMDD &        375.11 &             421.22 &          2.86 &              302.94 &       3872.57 &               0.59 &            9.47 &          1.91 &          0.33 \\
% \hline
%             LOCI &   \textit{TO} &        \textit{TO} &   \textit{TO} &         \textit{TO} &   \textit{TO} &              35.97 &     \textit{TO} &        154.49 &          6.84 \\
% \hline
%             LODA & \textbf{0.07} &      \textbf{0.10} & \textbf{0.04} &       \textbf{0.10} &          0.73 &      \textbf{0.03} &   \textbf{0.05} & \textbf{0.04} & \textbf{0.03} \\
% \hline
%             LOF &          3.59 &               5.49 & \textbf{0.02} &                1.50 &          1.36 &      \textbf{0.00} &   \textbf{0.06} & \textbf{0.01} & \textbf{0.00} \\
% \hline
%             MCD &         84.74 &               7.24 &          0.69 &                6.83 &         13.78 &      \textbf{0.05} &            0.81 &          0.11 & \textbf{0.05} \\
% \hline
%         MOGAAL &        267.93 &             415.02 &         40.89 &              463.18 &   \textit{TO} &              39.86 &           80.35 &         40.65 &         38.10 \\
% \hline
%             OCSVM &          2.51 &               3.71 & \textbf{0.03} &                3.14 &        252.52 &      \textbf{0.00} &            0.11 & \textbf{0.01} & \textbf{0.00} \\
% \hline
%             SOD &         56.94 &             131.49 &          2.48 &              210.39 &   \textit{TO} &               0.54 &           13.92 &          1.03 &          0.19 \\
% \hline
%         SOGAAL &         29.55 &              44.72 &          5.20 &               48.79 &        592.27 &               4.69 &           10.38 &          4.76 &          4.01 \\
% \hline
%             SOS &          9.37 &              26.09 &          1.01 &               34.96 &   \textit{TO} &               0.21 &            2.81 &          0.36 & \textbf{0.10} \\
% \hline
%             VAE &         30.38 &              30.58 &          5.67 &               30.41 &        176.84 &               4.03 &           10.46 &          5.33 &          4.62 \\
% \hline
% \end{tabular}
% \end{table*}


For the following table(s), we include comparisons against methods for which we could find a usable implementation.
For these algorithms, we compare the performance of CHAODA against that reported by the respective authors.

% TODO: Fill this in

\begin{table*}[!b]
\renewcommand{\arraystretch}{1.25}
\caption{Performance against misc. algorithms}
\label{table:results:misc-comparisons}
\begin{tabular}{|c|c|c|c|}
\hline
\textbf{model} & \textbf{dataset-1} & \textbf{dataset-2} & \textbf{dataset-3} \\
\hline
Misc-1 & 0.00 & 0.00 & 0.00 \\
\hline
Misc-2 & 0.00 & 0.00 & 0.00 \\
\hline
Misc-3 & 0.00 & 0.00 & 0.00 \\
\hline
\end{tabular}
\end{table*}

% TODO: Add table(s) against algorithms for which we did not find usable implementations


% TODO: Add the following tables to the supplementary materials.


% \subsection{New CHAODA Results}
% \label{subsec:results:new-chaoda-results}

% \begin{table*}[!b]
%     \renewcommand{\arraystretch}{1.25}
%     \caption{New CHAODA Results. Set 1}
%     \label{table:results:new-chaoda-1}
%     \centering
%     \begin{tabular}{|c|c|c|c|c|c|c|c|c|c|}
%         \hline
%         \textbf{voting} & \textbf{normed} & \textbf{\textbf{annthyroid}} & \textbf{\textbf{arrhythmia}} & \textbf{\textbf{breastw}} & \textbf{\textbf{cardio}} & \textbf{\textbf{cover}} & \textbf{\textbf{glass}} & \textbf{\textbf{http}} & \textbf{\textbf{ionosphere}} \\
%         \hline
%                    mean &            None &                \textbf{0.64} &                \textbf{0.77} &             \textbf{0.97} &            \textbf{0.81} &                    0.71 &                    0.70 &          \textbf{1.00} &                \textbf{0.88} \\
%         \hline
%                 product &            None &                         0.62 &                \textbf{0.77} &                      0.92 &                     0.76 &                    0.71 &                    0.63 &          \textbf{1.00} &                         0.80 \\
%         \hline
%                  median &            None &                         0.51 &                         0.57 &                      0.95 &                     0.55 &                    0.49 &                    0.53 &          \textbf{1.00} &                         0.60 \\
%         \hline
%                     min &            None &                         0.61 &                         0.73 &             \textbf{0.98} &            \textbf{0.79} &                    0.71 &                    0.69 &          \textbf{1.00} &                \textbf{0.87} \\
%         \hline
%                     max &            None &                         0.62 &                \textbf{0.75} &             \textbf{0.96} &            \textbf{0.81} &                    0.73 &           \textbf{0.73} &          \textbf{1.00} &                \textbf{0.88} \\
%         \hline
%                     p25 &            None &                \textbf{0.65} &                \textbf{0.75} &             \textbf{0.96} &            \textbf{0.80} &                    0.45 &                    0.70 &          \textbf{1.00} &                \textbf{0.87} \\
%         \hline
%                     p75 &            None &                         0.51 &                         0.74 &             \textbf{0.98} &                     0.73 &           \textbf{0.77} &           \textbf{0.75} &          \textbf{1.00} &                         0.61 \\
%         \hline
%                    mean &          linear &                         0.58 &                \textbf{0.75} &             \textbf{0.98} &            \textbf{0.79} &                    0.74 &           \textbf{0.73} &          \textbf{1.00} &                         0.85 \\
%         \hline
%                 product &          linear &                         0.61 &                         0.73 &             \textbf{0.98} &            \textbf{0.79} &                    0.71 &                    0.69 &          \textbf{1.00} &                \textbf{0.87} \\
%         \hline
%                  median &          linear &                         0.62 &                \textbf{0.75} &             \textbf{0.96} &            \textbf{0.81} &                    0.73 &           \textbf{0.73} &          \textbf{1.00} &                \textbf{0.88} \\
%         \hline
%                     min &          linear &                \textbf{0.65} &                \textbf{0.75} &             \textbf{0.96} &            \textbf{0.80} &                    0.45 &                    0.70 &          \textbf{1.00} &                \textbf{0.87} \\
%         \hline
%                     max &          linear &                         0.51 &                         0.74 &             \textbf{0.98} &                     0.73 &           \textbf{0.77} &           \textbf{0.75} &          \textbf{1.00} &                         0.61 \\
%         \hline
%                     p25 &          linear &                \textbf{0.64} &                \textbf{0.77} &             \textbf{0.97} &            \textbf{0.81} &                    0.71 &                    0.70 &          \textbf{1.00} &                \textbf{0.88} \\
%         \hline
%                     p75 &          linear &                         0.62 &                \textbf{0.77} &                      0.92 &                     0.76 &                    0.71 &                    0.63 &          \textbf{1.00} &                         0.80 \\
%         \hline
%                    mean &        gaussian &                         0.62 &                \textbf{0.75} &             \textbf{0.96} &            \textbf{0.81} &                    0.73 &           \textbf{0.73} &          \textbf{1.00} &                \textbf{0.88} \\
%         \hline
%                 product &        gaussian &                \textbf{0.65} &                \textbf{0.75} &             \textbf{0.96} &            \textbf{0.80} &                    0.45 &                    0.70 &          \textbf{1.00} &                \textbf{0.87} \\
%         \hline
%                  median &        gaussian &                         0.51 &                         0.70 &             \textbf{0.98} &                     0.73 &           \textbf{0.77} &           \textbf{0.75} &          \textbf{1.00} &                         0.61 \\
%         \hline
%                     min &        gaussian &                \textbf{0.64} &                \textbf{0.77} &             \textbf{0.97} &            \textbf{0.81} &                    0.71 &                    0.70 &          \textbf{1.00} &                \textbf{0.88} \\
%         \hline
%                     max &        gaussian &                         0.62 &                \textbf{0.77} &                      0.92 &                     0.76 &                    0.71 &                    0.63 &          \textbf{1.00} &                         0.80 \\
%         \hline
%                     p25 &        gaussian &                         0.54 &                         0.64 &                      0.50 &                     0.73 &                    0.50 &                    0.56 &          \textbf{1.00} &                         0.70 \\
%         \hline
%                     p75 &        gaussian &                         0.61 &                         0.73 &             \textbf{0.98} &            \textbf{0.79} &                    0.71 &                    0.69 &          \textbf{1.00} &                \textbf{0.87} \\
%         \hline
%                    mean &         sigmoid &                         0.51 &                         0.74 &             \textbf{0.98} &                     0.73 &           \textbf{0.77} &           \textbf{0.75} &          \textbf{1.00} &                         0.61 \\
%         \hline
%                 product &         sigmoid &                \textbf{0.64} &                \textbf{0.77} &             \textbf{0.97} &            \textbf{0.81} &                    0.71 &                    0.70 &          \textbf{1.00} &                \textbf{0.88} \\
%         \hline
%                  median &         sigmoid &                         0.62 &                \textbf{0.77} &                      0.92 &                     0.76 &                    0.71 &                    0.63 &          \textbf{1.00} &                         0.80 \\
%         \hline
%                     min &         sigmoid &                         0.54 &                         0.66 &                      0.50 &                     0.73 &                    0.50 &                    0.56 &          \textbf{1.00} &                         0.69 \\
%         \hline
%                     max &         sigmoid &                         0.61 &                         0.73 &             \textbf{0.98} &            \textbf{0.79} &                    0.71 &                    0.69 &          \textbf{1.00} &                \textbf{0.87} \\
%         \hline
%                     p25 &         sigmoid &                         0.62 &                \textbf{0.75} &             \textbf{0.96} &            \textbf{0.81} &                    0.73 &           \textbf{0.73} &          \textbf{1.00} &                \textbf{0.88} \\
%         \hline
%                     p75 &         sigmoid &                \textbf{0.65} &                \textbf{0.75} &             \textbf{0.96} &            \textbf{0.80} &                    0.45 &                    0.70 &          \textbf{1.00} &                \textbf{0.87} \\
%         \hline
%     \end{tabular}
% \end{table*}


% \begin{table*}[!b]
%     \renewcommand{\arraystretch}{1.25}
%     \caption{New CHAODA Results. Set 2}
%     \label{table:results:new-chaoda-2}
%     \centering
%     \begin{tabular}{|c|c|c|c|c|c|c|c|c|c|}
%         \hline
%         \textbf{voting} & \textbf{normed} & \textbf{\textbf{lympho}} & \textbf{\textbf{mammography}} & \textbf{\textbf{mnist}} & \textbf{\textbf{musk}} & \textbf{\textbf{optdigits}} & \textbf{\textbf{pendigits}} & \textbf{\textbf{pima}} & \textbf{\textbf{satellite}} \\
%         \hline
%                    mean &            None &            \textbf{0.98} &                 \textbf{0.85} &           \textbf{0.78} &          \textbf{1.00} &               \textbf{0.96} &               \textbf{0.94} &                   0.63 &               \textbf{0.79} \\
%         \hline
%                 product &            None &                     0.92 &                          0.80 &                    0.71 &          \textbf{1.00} &                        0.88 &                        0.88 &                   0.60 &                        0.76 \\
%         \hline
%                  median &            None &                     0.91 &                          0.52 &                    0.56 &          \textbf{1.00} &                        0.81 &                        0.69 &                   0.52 &                        0.67 \\
%         \hline
%                     min &            None &            \textbf{0.97} &                          0.83 &                    0.76 &          \textbf{1.00} &               \textbf{0.95} &               \textbf{0.95} &                   0.64 &               \textbf{0.79} \\
%         \hline
%                     max &            None &            \textbf{0.98} &                 \textbf{0.86} &                    0.75 &          \textbf{1.00} &                        0.88 &               \textbf{0.94} &                   0.63 &               \textbf{0.79} \\
%         \hline
%                     p25 &            None &                     0.95 &                 \textbf{0.85} &                    0.66 &          \textbf{1.00} &                        0.91 &               \textbf{0.94} &                   0.62 &               \textbf{0.77} \\
%         \hline
%                     p75 &            None &                     0.94 &                          0.83 &                    0.71 &          \textbf{0.99} &                        0.82 &                        0.64 &                   0.60 &                        0.72 \\
%         \hline
%                    mean &          linear &            \textbf{0.98} &                 \textbf{0.85} &           \textbf{0.79} &          \textbf{1.00} &                        0.93 &                        0.77 &          \textbf{0.67} &               \textbf{0.79} \\
%         \hline
%                 product &          linear &            \textbf{0.97} &                          0.83 &                    0.76 &          \textbf{1.00} &               \textbf{0.95} &               \textbf{0.95} &                   0.64 &               \textbf{0.79} \\
%         \hline
%                  median &          linear &            \textbf{0.98} &                 \textbf{0.86} &                    0.75 &          \textbf{1.00} &                        0.88 &               \textbf{0.94} &                   0.63 &               \textbf{0.79} \\
%         \hline
%                     min &          linear &                     0.95 &                 \textbf{0.85} &                    0.66 &          \textbf{1.00} &                        0.91 &               \textbf{0.94} &                   0.62 &               \textbf{0.77} \\
%         \hline
%                     max &          linear &                     0.94 &                          0.83 &                    0.71 &          \textbf{0.99} &                        0.82 &                        0.64 &                   0.60 &                        0.72 \\
%         \hline
%                     p25 &          linear &            \textbf{0.98} &                 \textbf{0.85} &           \textbf{0.78} &          \textbf{1.00} &               \textbf{0.96} &               \textbf{0.94} &                   0.63 &               \textbf{0.79} \\
%         \hline
%                     p75 &          linear &                     0.92 &                          0.80 &                    0.71 &          \textbf{1.00} &                        0.88 &                        0.88 &                   0.60 &                        0.76 \\
%         \hline
%                    mean &        gaussian &            \textbf{0.98} &                 \textbf{0.86} &                    0.75 &          \textbf{1.00} &                        0.88 &               \textbf{0.94} &                   0.63 &               \textbf{0.79} \\
%         \hline
%                 product &        gaussian &                     0.95 &                 \textbf{0.85} &                    0.66 &          \textbf{1.00} &                        0.91 &               \textbf{0.94} &                   0.62 &               \textbf{0.77} \\
%         \hline
%                  median &        gaussian &                     0.94 &                          0.83 &                    0.71 &          \textbf{0.99} &                        0.82 &                        0.65 &                   0.60 &                        0.72 \\
%         \hline
%                     min &        gaussian &            \textbf{0.98} &                 \textbf{0.85} &           \textbf{0.78} &          \textbf{1.00} &               \textbf{0.96} &               \textbf{0.94} &                   0.63 &               \textbf{0.79} \\
%         \hline
%                     max &        gaussian &                     0.92 &                          0.80 &                    0.71 &          \textbf{1.00} &                        0.88 &                        0.88 &                   0.60 &                        0.76 \\
%         \hline
%                     p25 &        gaussian &            \textbf{0.98} &                          0.60 &                    0.69 &          \textbf{1.00} &                        0.50 &                        0.50 &                   0.57 &                        0.50 \\
%         \hline
%                     p75 &        gaussian &            \textbf{0.97} &                          0.83 &                    0.76 &          \textbf{1.00} &               \textbf{0.95} &               \textbf{0.95} &                   0.64 &               \textbf{0.79} \\
%         \hline
%                    mean &         sigmoid &                     0.94 &                          0.83 &                    0.71 &          \textbf{0.99} &                        0.82 &                        0.64 &                   0.60 &                        0.72 \\
%         \hline
%                 product &         sigmoid &            \textbf{0.98} &                 \textbf{0.85} &           \textbf{0.78} &          \textbf{1.00} &               \textbf{0.96} &               \textbf{0.94} &                   0.63 &               \textbf{0.79} \\
%         \hline
%                  median &         sigmoid &                     0.92 &                          0.80 &                    0.71 &          \textbf{1.00} &                        0.88 &                        0.88 &                   0.60 &                        0.76 \\
%         \hline
%                     min &         sigmoid &            \textbf{0.98} &                          0.60 &                    0.70 &          \textbf{1.00} &                        0.50 &                        0.50 &                   0.56 &                        0.50 \\
%         \hline
%                     max &         sigmoid &            \textbf{0.97} &                          0.83 &                    0.76 &          \textbf{1.00} &               \textbf{0.95} &               \textbf{0.95} &                   0.64 &               \textbf{0.79} \\
%         \hline
%                     p25 &         sigmoid &            \textbf{0.98} &                 \textbf{0.86} &                    0.75 &          \textbf{1.00} &                        0.88 &               \textbf{0.94} &                   0.63 &               \textbf{0.79} \\
%         \hline
%                     p75 &         sigmoid &                     0.95 &                 \textbf{0.85} &                    0.66 &          \textbf{1.00} &                        0.91 &               \textbf{0.94} &                   0.62 &               \textbf{0.77} \\
%         \hline
%     \end{tabular}              
% \end{table*}


% \begin{table*}[!b]
%     \renewcommand{\arraystretch}{1.25}
%     \caption{New CHAODA Results. Set 3}
%     \label{table:results:new-chaoda-3}
%     \centering
%     \begin{tabular}{|c|c|c|c|c|c|c|c|c|c|}
%         \hline
%         \textbf{voting} & \textbf{normed} & \textbf{\textbf{satimage-2}} & \textbf{\textbf{shuttle}} & \textbf{\textbf{smtp}} & \textbf{\textbf{thyroid}} & \textbf{\textbf{vertebral}} & \textbf{\textbf{vowels}} & \textbf{\textbf{wbc}} & \textbf{\textbf{wine}} \\
%         \hline
%                     mean &            None &                \textbf{1.00} &             \textbf{0.51} &                   0.92 &             \textbf{0.89} &                        0.29 &                     0.71 &         \textbf{0.97} &          \textbf{0.99} \\
%         \hline
%                 product &            None &                         0.95 &             \textbf{0.51} &                   0.45 &                      0.75 &                        0.36 &                     0.59 &         \textbf{0.97} &          \textbf{0.98} \\
%         \hline
%                     median &            None &                         0.93 &             \textbf{0.50} &                   0.84 &                      0.56 &               \textbf{0.46} &                     0.52 &                  0.82 &          \textbf{0.99} \\
%         \hline
%                     min &            None &                \textbf{0.98} &             \textbf{0.51} &                   0.91 &                      0.83 &                        0.32 &                     0.67 &                  0.95 &          \textbf{0.98} \\
%         \hline
%                     max &            None &                \textbf{0.99} &             \textbf{0.51} &          \textbf{0.95} &             \textbf{0.90} &                        0.30 &                     0.70 &         \textbf{0.97} &          \textbf{1.00} \\
%         \hline
%                     p25 &            None &                \textbf{0.99} &             \textbf{0.51} &                   0.87 &             \textbf{0.90} &                        0.29 &                     0.69 &         \textbf{0.98} &          \textbf{1.00} \\
%         \hline
%                     p75 &            None &                         0.96 &             \textbf{0.51} &                   0.87 &                      0.62 &               \textbf{0.47} &                     0.72 &                  0.89 &          \textbf{0.99} \\
%         \hline
%                     mean &          linear &                \textbf{0.99} &             \textbf{0.51} &                   0.91 &             \textbf{0.88} &                        0.36 &            \textbf{0.79} &                  0.95 &          \textbf{0.99} \\
%         \hline
%                 product &          linear &                \textbf{0.98} &             \textbf{0.51} &                   0.91 &                      0.83 &                        0.32 &                     0.67 &                  0.95 &          \textbf{0.98} \\
%         \hline
%                     median &          linear &                \textbf{0.99} &             \textbf{0.51} &          \textbf{0.95} &             \textbf{0.90} &                        0.30 &                     0.70 &         \textbf{0.97} &          \textbf{1.00} \\
%         \hline
%                     min &          linear &                \textbf{0.99} &             \textbf{0.51} &                   0.87 &             \textbf{0.90} &                        0.29 &                     0.69 &         \textbf{0.98} &          \textbf{1.00} \\
%         \hline
%                     max &          linear &                         0.96 &             \textbf{0.51} &                   0.87 &                      0.62 &               \textbf{0.47} &                     0.72 &                  0.89 &          \textbf{0.99} \\
%         \hline
%                     p25 &          linear &                \textbf{1.00} &             \textbf{0.51} &                   0.92 &             \textbf{0.89} &                        0.29 &                     0.71 &         \textbf{0.97} &          \textbf{0.99} \\
%         \hline
%                     p75 &          linear &                         0.95 &             \textbf{0.51} &                   0.45 &                      0.75 &                        0.36 &                     0.59 &         \textbf{0.97} &          \textbf{0.98} \\
%         \hline
%                     mean &        gaussian &                \textbf{0.99} &             \textbf{0.51} &          \textbf{0.95} &             \textbf{0.90} &                        0.30 &                     0.70 &         \textbf{0.97} &          \textbf{1.00} \\
%         \hline
%                 product &        gaussian &                \textbf{0.99} &             \textbf{0.51} &                   0.87 &             \textbf{0.90} &                        0.29 &                     0.69 &         \textbf{0.98} &          \textbf{1.00} \\
%         \hline
%                     median &        gaussian &                         0.96 &             \textbf{0.51} &                   0.87 &                      0.62 &               \textbf{0.47} &                     0.72 &                  0.89 &          \textbf{0.99} \\
%         \hline
%                     min &        gaussian &                \textbf{1.00} &             \textbf{0.51} &                   0.92 &             \textbf{0.89} &                        0.29 &                     0.71 &         \textbf{0.97} &          \textbf{0.99} \\
%         \hline
%                     max &        gaussian &                         0.95 &             \textbf{0.51} &                   0.45 &                      0.75 &                        0.36 &                     0.59 &         \textbf{0.97} &          \textbf{0.98} \\
%         \hline
%                     p25 &        gaussian &                \textbf{0.98} &             \textbf{0.50} &                   0.50 &                      0.63 &               \textbf{0.46} &                     0.50 &                  0.91 &          \textbf{0.99} \\
%         \hline
%                     p75 &        gaussian &                \textbf{0.98} &             \textbf{0.51} &                   0.91 &                      0.83 &                        0.32 &                     0.67 &                  0.95 &          \textbf{0.98} \\
%         \hline
%                     mean &         sigmoid &                         0.96 &             \textbf{0.51} &                   0.87 &                      0.62 &               \textbf{0.47} &                     0.72 &                  0.89 &          \textbf{0.99} \\
%         \hline
%                 product &         sigmoid &                \textbf{1.00} &             \textbf{0.51} &                   0.92 &             \textbf{0.89} &                        0.29 &                     0.71 &         \textbf{0.97} &          \textbf{0.99} \\
%         \hline
%                     median &         sigmoid &                         0.95 &             \textbf{0.51} &                   0.45 &                      0.75 &                        0.36 &                     0.59 &         \textbf{0.97} &          \textbf{0.98} \\
%         \hline
%                     min &         sigmoid &                \textbf{0.98} &             \textbf{0.50} &                   0.50 &                      0.62 &               \textbf{0.46} &                     0.50 &                  0.91 &          \textbf{0.99} \\
%         \hline
%                     max &         sigmoid &                \textbf{0.98} &             \textbf{0.51} &                   0.91 &                      0.83 &                        0.32 &                     0.67 &                  0.95 &          \textbf{0.98} \\
%         \hline
%                     p25 &         sigmoid &                \textbf{0.99} &             \textbf{0.51} &          \textbf{0.95} &             \textbf{0.90} &                        0.30 &                     0.70 &         \textbf{0.97} &          \textbf{1.00} \\
%         \hline
%                     p75 &         sigmoid &                \textbf{0.99} &             \textbf{0.51} &                   0.87 &             \textbf{0.90} &                        0.29 &                     0.69 &         \textbf{0.98} &          \textbf{1.00} \\
%         \hline
%     \end{tabular}   
% \end{table*}

    \section{Discussion}
\label{sec:discussion}

We have presented CHAODA, an ensemble of six algorithms that use the map of the underlying manifold produced by CLAM\@.
The six individual algorithms are simple to implement on top of CLAM and, when combined into an ensemble, often outperform state-of-the-art methods.

CLAM uses the geometric and topological properties of fractal dimension and metric entropy of the data to build a map of the low-dimensional manifold that the data occupy.
CLAM is inspired by the clustering algorithm from CHESS~\cite{ishaq2019clustered}, but differs by the introduction a novel graph-induction approach and a notion of optimal depths, learned via a form of ``meta-machine-learning'' and transfer learning.
CLAM's selection of ``poles'' for partitioning clusters is also more efficient than CHESS.
CHAODA builds on this manifold-mapping framework for anomaly and outlier detection.
Intuitively, we expect CHAODA to perform particularly well when the data lie on an ``interesting'' manifold, and to perform merely average when the data derive from an easily-described distribution (or ``boring'' manifold).
Just as CHESS demonstrated an acceleration of search when the data exhibited \emph{low fractal dimension} and \emph{low metric entropy}, with CHAODA, we see that AUC scores are vastly improved when the data exhibit these properties and often competitive when the data do not exhibit these properties.
We believe that CLAM is free of hyperparameters; the weights learned from the meta-ML step could vary, but we learned them once on a distinct training set.

We should note the Vertebral Column (Vert.) dataset.
Most of the tested algorithms perform similarly to random guessing, while CHAODA performs much worse.
We suspect this is because of how this specific dataset was collected.
Each instance represents 6 biomechanical attributes derived from scans of a patient's pelvis and lumbar spine.
% These attributes are: pelvic incidence, pelvic tilt, lumbar lordosis angle, sacral slope, pelvic radius and grade of spondylolisthesis.
This dataset contains 210 instances of the Abnormal class treated as inliers, and 30 instances of the Normal class treated as outliers.
Each attribute has a narrow range to be in the Normal class, but can have a wider range in the Abnormal class.
This causes the Normal instances to group together, while Abnormal instances remain distant from each other.
Since CHAODA relies on clusters as the substrate, they assign low scores to the instances in the Normal class, i.e.\ the outliers, and high scores to the Abnormal class, i.e.\ the inliers.
Put plainly, CHAODA sees the manifold as the Normal class, which the dataset treats as outliers.
This curiosity could be remedied by a supervised version of CHAODA.

% A simple visualization in Figure~\ref{fig:conclusions:umap-embeddings} using UMAP~\cite{mcinnes2018umap} illustrates two different examples; the anomalies in one dataset, where CHAODA outperforms other methods, appear to be at the edges of a complex manifold (though, clearly, the UMAP projection has distorted the manifold) while in another dataset, where most methods perform fairly comparably, the distribution is less interesting and many anomalies are distributed apparently randomly across it.

% \begin{figure*}
%    \centering
%    \includegraphics[width=2in]{images/umaps/mnist-euclidean-umap2d.png}
%    \includegraphics[width=2in]{images/umaps/ionosphere-euclidean-umap2d.png}
%    \caption{UMAP projection of MNIST (left) and Ionosphere (right). Anomalies are in gray. Note that for MNIST, the UMAP projection does not find much structure, though most of the anomalies congregate to one side. For Ionosphere, there is a single main component to the manifold, with two more distant clusters, and anomalies tend to be at the edges of a manifold. Most algorithms perform comparably on MNIST, while CHAODA outperforms others on Ionosphere.}
%    \label{fig:conclusions:umap-embeddings}
% \end{figure*}

% \subsection{Future Directions}
% \label{subsec:discussion:future-directions}


The choice of distance function could have a significant impact on anomaly-detection performance.
In this case, domain knowledge is likely the best way to determine the distance function of choice.
Future work should explore a more diverse collection of domain-appropriate distance functions, such as Wasserstein distance on images, Levenshtein edit distance on strings, and Jaccard distance on the maximal common subgraph of molecular structures.

CHAODA is demonstrably highly effective on high-dimensional datasets, and so may be applied to neural networks.
Using CLAM to map a dataset where each datum represents the activations of a neural network from an input to the neural network, we would expect to detect malicious inputs to neural networks based on the intuition that malicious inputs produce atypical activation patterns.

In conclusion, we have demonstrated that by mapping the manifolds occupied by data, CLAM reveals structure that allows CHAODA to outperform other state-of-the-art approaches to anomaly detection.

    %  Acknowledgments are not allowed in the blind review version
    % (damn! they're onto us!)
    % \section{Acknowledgements}
\label{sec:acknowledgements}

TODO

% The authors would like to thank the members of CSC 592 - Algorithms for Big Data for helpful feedback and discussions.
% We have been Clearly Hoping that Our Work Detects Anomalies.


    % \afterpage{\clearpage}
    \FloatBarrier
    \bibliographystyle{acm}
    \bibliography{references}
    \newpage
    \documentclass{article}


% \usepackage{arxiv} # uncomment for preprint

\usepackage[utf8]{inputenc} % allow utf-8 input
\usepackage[T1]{fontenc}    % use 8-bit T1 fonts
% \usepackage{hyperref}       % hyperlinks
\usepackage{url}            % simple URL typesetting
\usepackage{booktabs}       % professional-quality tables
\usepackage{amsfonts}       % blackboard math symbols
\usepackage{nicefrac}       % compact symbols for 1/2, etc.
\usepackage{microtype}      % microtypography
\usepackage{lipsum}
\usepackage{graphicx}
\usepackage{subfigure}
\usepackage{amsmath}

\begin{document}

\section*{Datasets}

The \textbf{annthyroid} dataset is derived from the ``Thyroid Disease'' dataset from the UCIMLR\@.
The original data has 7200 instances with 15 categorical attributes and 6 real-valued attributes.
The class labels are ``normal'', ``hypothyroid'', and ``subnormal''.
For anomaly detection, the ``hypothyroid'' and ``subnormal'' classes are combined into 534 outlier instances, and only the 6 real-valued attributes are used.

The \textbf{arrhythmia} dataset is derived from ``Arrhythmia`` dataset from the UCIMLR\@.
The original dataset contains 452 instances with 279 attributes.
There are five categorical attributes which are discarded, leaving this as a 274-dimensional dataset.
The instances are divided into 16 classes.
The eight smallest classes collectively contain 66 instances and are combined into the outlier class.

The \textbf{breastw} dataset is also derived from the ``Breast Cancer Wisconsin (Original)`` dataset.
This is a 9-dimensional dataset containing 683 instances of which 239 represent malignant tumors and are treated as the outlier class.

The \textbf{cardio} dataset is derived from the ``Cardiotocography'' dataset.
The dataset is composed of measurements of fetal heart rate and uterine contraction features on cardiotocograms.
The are each labelled ``normal'', ``suspect'', and ``pathologic'' by expert obstetricians.
For anomaly detection, the ``normal'' class forms the inliers, the ``suspect'' class is discarded, and the ``pathologic'' class is downsampled to 176 instances forming the outliers.
This leaves us with 1831 instances with 21 attributes in the dataset.

The \textbf{cover} dataset is derived from the ``Covertype'' dataset.
The original dataset contains 581,012 instances with 54 attributes.
The dataset is used to predict the type of forest cover solely from cartographic variables.
The instances are labelled into seven different classes.
For outlier detection, we use only the 10 quantitative attributes, type 2 (lodgepole pine) as the inliers, and type 4 (conttonwood/willow) as the outliers.
The remaining classes are discarded.
This leaves us with a 10-dimensional dataset with 286,048 instances of which 2,747 are outliers.

The \textbf{glass} dataset is derived from the ``Glass Identification'' dataset.
The study of classification of types of glass was motivated by criminological investigations where glass fragments left at crime scenes were used as evidence.
This dataset contains 214 instances with nine attributes.
While there are several different types of glass in this dataset, class 6 is a clear minority with only nine instances and, as such, points in class 6 are treated as the outliers while all other classes are treated as inliers.

The \textbf{http} dataset is derived from the original ``KDD Cup 1999'' dataset.
It contains 41 attributes (34 continuous and 7 categorical) which are reduced to 4 attributes (service, duration, src\_bytes, dst\_bytes).
Only the ``service'' attribute is categorical, dividing the data into {http, smtp, ftp, ftp\_data, others} subsets.
Here, only the ``http'' data is used.
The values of the continuous attributes are centered around 0, so they have been log-transformed far away from 0.
The original data contains 3,925,651 attacks in 4,898,431 records.
This smaller dataset is created with only 2,211 attacks in 567,479 records.

The \textbf{ionosphere} dataset is derived from the ``Ionosphere'' dataset.
It consists 351 instances with 34 attributes.
One of the attributes is always 0 and, so, is discarded, leaving us with a 33-dimensional dataset.
The data comes from radar measurements of the ionosphere from a system located in Goose Bay, Labrador.
The data are classified into ``good'' if the radar returns evidence some type of structure in the ionosphere, and ``bad'' if not.
The ``good'' class serves as the inliers and the ``bad'' class serves as the outliers.

The \textbf{lympho} dataset is derived from the ``Lymphography'' dataset.
The data contain 148 instances with 18 attributes.
The instances are labelled ``normal find'', ``metastases'', ``malign lymph'', and ``fibrosis''.
The two minority classes only contain a total of six instances, and are combined to form the outliers.
The remaining 142 instances form the inliers.

The \textbf{mammography} dataset is derived from the original ``Mammography'' dataset provided by Aleksandar Lazarevic.
Its goal is to use x-ray images of human breasts to find calcified tissue as an early sign of breast cancer.
As such, the ``calcification'' class is considered as the outlier class while the ``non-clacification'' class is the inliers.
We have 11,183 instances with 6 attributes, of which 260 are ``calcifications.''

The \textbf{mnist} dataset is derived from the classic ``MNIST'' dataset of handwritten digits.
Digit-zero is considered the inlier class while 700 images of digit-six are the outliers.
Furthermore, 100 pixels are randomly selected as features from the original 784 pixels.

The \textbf{musk} dataset is derived from its namesake in the UCIMLR\@.
It is created from molecules that have been classified by experts as ``musk'' or ``non-musk''.
The data are downsampled to 3,062 instances with 166 attributes.
The ``musk'' class forms the outliers while the ``non-musk'' class forms the inliers.

The \textbf{optdigits} dataset is derived from the ``Optical Recognition of Handwritten Digits'' dataset.
Digits 1--9 form the inliers while 150 samples of digit-zero form the outliers.
This gives us a dataset of 5,216 instances with 64 attributes.

The \textbf{pendigits} dataset is derived from the ``Pen-Based Recognition of Handwritten Digits'' dataset from the UCI Machine Learning Repository.
The original collection of handwritten samples is reduced to 6,870 points, of which 156 are outliers.

The \textbf{pima} dataset is derived from the ``Pima Indians Diabetes'' dataset.
The original dataset presents a binary classification problem to detect diabetes.
This subset was restricted to female patients at least 21 years old of Pima Indian heritage.

The \textbf{satellite} dataset is derived from the ``Statlog (Landsat Satellite)'' dataset.
The smallest three classes (2, 4, and 5) are combined to form the outlier class while the other classes are combined to form the inlier class.
The train and test subsets are combined to produce a of 6,435 instances with 36 attributes.

The \textbf{satimage-2} dataset is also derived from the ``Satlog (Landsat Satellite)'' dataset.
Class 2 is downsampled to 71 instances that are treated as outliers, while all other classes are combined to form an inlier class.
This gives us 5,803 instances with 36 attributes.

The \textbf{shuttle} dataset is derived from the ``Statlog (Shuttle)'' dataset.
This are seven classes in the original dataset.
Here, class 4 is discarded, class 1 is treated as the inliers and the remaining classes, which are comparatively small, form an outlier class.
This gives us 49,097 instances with 9 attributes, of which 3,511 are outliers.

The \textbf{smtp} is also derived from the ``KDD Cup 1999'' dataset.
It is preprocessed in the same way as the \textbf{http} dataset, except that the ``smtp'' service subset is used.
This version of the dataset only contains 95,156 instances with 3 attributes, of which 30 instances are outliers.

The \textbf{thyroid} dataset is also derived from the ``Thyroid Disease'' dataset.
The attribute selection is the same as for the \textbf{annthyroid} dataset but only the 3,772 training instances are used in this version.
The ``hyperfunction'' class, containing 93 instances, is treated as the outlier class, while the other two classes are combined to form an inlier class.

The \textbf{vertebral} dataset is derived from the ``Vertebral Column'' dataset.
6 attributes are derived to represent the shape and orientation of the pelvis and lumbar spine.
Each instance comes from a different patient.
The ``Abnormal (AB)'' class of 210 instances are used as inliers while the ``Normal (NO)'' class is downsampled to 30 instances to be used as outliers.

The \textbf{vowels} dataset is derived from the ``Japanese Vowels'' dataset.
THE UCIMLR presents this data as a multivariate time series of nine speakers uttering two Japanese vowels.
For outlier detection, each frame of each time-series is treated as a separate point.
There are 12 features associated with each time series, and these translate as the attributes for each point.
Data from speaker 1, downsampled to 50 points, form the outlier class/
Speakers 6, 7, and 8 form the outlier class.
The rest of the points are discarded.
This leaves is with 1,456 points in 12 dimensions, of which 50 are outliers.

The \textbf{wbc} dataset is derived from the ``Wisconsin-Breast Cancer (Diagnostics)'' dataset.
The dataset records measurements for breast cancer cases.
The benign class is treated as the inlier class, while the malignant class is downsampled to 21 points and serves as the outlier class.
This leaves us with 278 points in 30 dimensions.

The \textbf{wine} dataset is a collection of results of a chemical analysis of several wines from a region in Italy.
The data contain 129 samples having 13 attributes, and divided into 3 classes.
Classes 2 and 3 form the inliers while class 1, downsampled to 10 instances, is the outlier class.


\begin{table*}[!t]
    \renewcommand{\arraystretch}{1.25}
    \caption{Datasets used for Benchmarks}
    \label{table:methods:benchmarks}
    \centering
    \begin{tabular}{|c|c|c|c|c|}
    \hline
    \textbf{Dataset} & \textbf{Cardinality} & \textbf{\# Dim.} & \textbf{\# Outliers} & \textbf{\% Outliers} \\
    \hline
    annthyroid & 7,200 & 6 & 534 & 7.42 \\
    \hline
    arrhythmia & 452 & 274 & 66 & 15 \\
    \hline
    breastw & 683 & 9 & 239 & 35 \\
    \hline
    cardio & 1,831 & 21 & 176 & 9.6 \\
    \hline
    cover & 286,048 & 10 & 2,747 & 0.9 \\
    \hline
    glass & 214 & 9 & 9 & 4.2 \\
    \hline
    http & 567,479 & 4 & 2,211 & 0.4 \\
    \hline
    ionosphere & 351 & 33 & 126 & 36 \\
    \hline
    lympho & 148 & 18 & 6 & 4.1 \\
    \hline
    mammography & 11,183 & 6 & 260 & 2.32 \\
    \hline
    mnist & 7603 & 100 & 700 & 9.2 \\
    \hline
    musk & 3,062 & 166 & 97 & 3.2 \\
    \hline
    optdigits & 5,216 & 64 & 150 & 3 \\
    \hline
    pendigits & 6,870 & 16 & 156 & 2.27 \\
    \hline
    pima & 768 & 8 & 268 & 35 \\
    \hline
    satellite & 6,435 & 36 & 2036 & 32 \\
    \hline
    satimage-2 & 5,803 & 36 & 71 & 1.2 \\
    \hline
    shuttle & 59,097 & 9 & 3,511 & 7 \\
    \hline
    smtp & 95,156 & 3 & 30 & 0.03 \\
    \hline
    thyroid & 3,772 & 6 & 93 & 2.5 \\
    \hline
    vertebral & 240 & 6 & 30 & 12.5 \\
    \hline
    vowels & 1,456 & 12 & 50 & 3.4 \\
    \hline
    wbc & 278 & 30 & 21 & 5.6 \\
    \hline
    wine & 129 & 13 & 10 & 7.7 \\
    \hline
    \end{tabular}
    \end{table*}

% TODO: Add to supplements

% \begin{table*}[!t]
% \renewcommand{\arraystretch}{1.25}
% \caption{Time taken, in seconds, on Train Datasets}
% \label{table:results:train-time}
% \centering
% \begin{tabular}{|c|c|c|c|c|c|c|}
% \hline
% \textbf{model} & \textbf{annthyroid} & \textbf{mnist} & \textbf{pendigits} & \textbf{satellite} & \textbf{shuttle} & \textbf{thyroid} \\
% \hline
%         CHAODA-fast &              161.85 &        6221.64 &             617.93 &             425.03 &             5.48 &           117.97 \\
% \hline
%         CHAODA &              205.74 &       10152.47 &             771.89 &             891.19 &         21415.56 &           219.32 \\
% \hline
%                 ABOD &                3.42 &          16.39 &               2.83 &               3.53 &            21.58 &             1.25 \\
% \hline
%         AutoEncoder &               22.80 &          40.54 &              23.87 &              26.74 &           158.72 &            12.84 \\
% \hline
%                 CBLOF &                1.01 &           1.14 &               0.23 &               0.30 &             0.71 &             0.21 \\
% \hline
%                 COF &              215.63 &         277.91 &             199.03 &             176.35 &         12106.63 &            54.82 \\
% \hline
%                 HBOS &                1.58 &  \textbf{0.06} &      \textbf{0.01} &      \textbf{0.02} &    \textbf{0.03} &    \textbf{0.00} \\
% \hline
%         IFOREST &                0.73 &           1.41 &               0.80 &               0.84 &             3.24 &             0.54 \\
% \hline
%                 KNN &                0.84 &          14.30 &               1.25 &               2.05 &             9.13 &             0.42 \\
% \hline
%                 LMDD &              107.05 &        1383.19 &             197.48 &             371.63 &          6274.43 &            32.69 \\
% \hline
%                 LOCI &         \textit{TO} &    \textit{TO} &        \textit{TO} &        \textit{TO} &      \textit{TO} &      \textit{TO} \\
% \hline
%                 LODA &                0.13 &           0.13 &               0.12 &               0.11 &             0.59 &    \textbf{0.08} \\
% \hline
%                 LOF &                0.26 &          14.79 &               0.90 &               1.71 &             6.30 &    \textbf{0.09} \\
% \hline
%                 MCD &                1.50 &          20.56 &               3.01 &              10.39 &            10.78 &             1.03 \\
% \hline
%         MOGAAL &              594.69 &    \textit{TO} &             561.40 &             513.84 &      \textit{TO} &           280.80 \\
% \hline
%                 OCSVM &                2.70 &          11.30 &               3.04 &               3.83 &           162.18 &             0.68 \\
% \hline
%                 SOD &              262.80 &         281.86 &             195.81 &             253.71 &      \textit{TO} &            91.48 \\
% \hline
%         SOGAAL &               61.14 &          68.03 &              55.40 &              50.07 &           460.66 &            31.43 \\
% \hline
%                 SOS &              101.82 &          54.22 &              46.02 &              42.62 &      \textit{TO} &            42.10 \\
% \hline
%                 VAE &               27.95 &          50.16 &              29.63 &              33.51 &           182.23 &            16.18 \\
% \hline
% \end{tabular}
% \end{table*}

% TODO: Add to supplements

% \begin{table*}[!t]
% \renewcommand{\arraystretch}{1.25}
% \caption{Time taken, in seconds, on the first half of the Test Datasets}
% \label{table:results:test-time-1}
% \centering
% \begin{tabular}{|c|c|c|c|c|c|c|c|c|c|}
% \hline
% \textbf{model} & \textbf{arrhythmia} & \textbf{breastw} & \textbf{cardio} & \textbf{cover} & \textbf{glass} & \textbf{http} & \textbf{ionosphere} & \textbf{lympho} & \textbf{mammo} \\
% \hline
%         CHAODA-fast &                6.70 &             4.71 &          135.17 &        1485.24 &           1.06 &        119.14 &                2.64 &            1.08 &          46.42 \\
% \hline
%         CHAODA &                6.90 &             4.80 &          143.31 &        3118.26 &           1.10 &        221.81 &                2.78 &            1.74 &          71.54 \\
% \hline
%                 ABOD &                0.34 &             0.20 &            0.72 &          24.02 &  \textbf{0.07} &         19.08 &                0.13 &   \textbf{0.05} &           3.82 \\
% \hline
%         AutoEncoder &                8.00 &             6.18 &            9.05 &         183.70 &           3.99 &        154.20 &                4.99 &            3.79 &          35.26 \\
% \hline
%                 CBLOF &                0.16 &             0.13 &            0.17 &    \textit{EX} &  \textbf{0.06} &   \textit{EX} &       \textbf{0.07} &   \textbf{0.05} &           0.20 \\
% \hline
%                 COF &                1.24 &             1.84 &           12.63 &       21681.15 &           0.24 &      21228.44 &                0.60 &            0.14 &         513.46 \\
% \hline
%                 HBOS &       \textbf{0.08} &    \textbf{0.00} &   \textbf{0.01} &           1.13 &  \textbf{0.00} & \textbf{0.01} &       \textbf{0.01} &   \textbf{0.01} &  \textbf{0.01} \\
% \hline
%         IFOREST &                0.43 &             0.34 &            0.43 &           4.61 &           0.30 &          3.95 &                0.33 &            0.30 &           0.95 \\
% \hline
%                 KNN &                0.19 &    \textbf{0.07} &            0.30 &          11.46 &  \textbf{0.02} &          7.47 &       \textbf{0.05} &   \textbf{0.02} &           1.58 \\
% \hline
%                 LMDD &               14.85 &             2.62 &           22.12 &       11579.93 &           0.57 &       3873.99 &                1.73 &            0.42 &         243.01 \\
% \hline
%                 LOCI &              307.04 &      \textit{TO} &     \textit{TO} &    \textit{TO} &          25.68 &   \textit{TO} &              120.92 &            9.78 &    \textit{TO} \\
% \hline
%                 LODA &       \textbf{0.04} &    \textbf{0.03} &   \textbf{0.05} &           0.81 &  \textbf{0.03} &          0.66 &       \textbf{0.03} &   \textbf{0.03} &           0.17 \\
% \hline
%                 LOF &                0.16 &    \textbf{0.01} &            0.18 &          10.24 &  \textbf{0.00} &          1.93 &       \textbf{0.01} &   \textbf{0.00} &           0.59 \\
% \hline
%                 MCD &                5.08 &             0.64 &            0.89 &          28.65 &  \textbf{0.05} &          8.21 &                0.15 &   \textbf{0.05} &           2.11 \\
% \hline
%         MOGAAL &               46.46 &            42.09 &          116.73 &    \textit{TO} &          40.67 &   \textit{TO} &               40.86 &           37.84 &    \textit{TO} \\
% \hline
%                 OCSVM &       \textbf{0.10} &    \textbf{0.02} &            0.23 &         257.95 &  \textbf{0.00} &        257.28 &       \textbf{0.01} &   \textbf{0.00} &           6.37 \\
% \hline
%                 SOD &                1.19 &             1.89 &           13.39 &    \textit{TO} &           0.31 &   \textit{TO} &                0.76 &            0.20 &         521.25 \\
% \hline
%         SOGAAL &                6.70 &             5.03 &           14.43 &         591.14 &           3.95 &        597.17 &                4.71 &            3.96 &          92.27 \\
% \hline
%                 SOS &                0.69 &            47.40 &            4.11 &    \textit{TO} &           0.18 &   \textit{TO} &                0.33 &            0.11 &    \textit{TO} \\
% \hline
%                 VAE &               10.00 &             7.75 &           10.89 &         223.09 &           5.27 &        175.59 &                6.66 &            5.21 &          40.87 \\
% \hline
% \end{tabular}
% \end{table*}

% TODO: Add to supplements

% \begin{table*}[!b]
% \renewcommand{\arraystretch}{1.25}
% \caption{Time taken, in seconds, on the second half of the Test Datasets}
% \label{table:results:test-time-2}
% \centering
% \begin{tabular}{|c|c|c|c|c|c|c|c|c|c|}
% \hline
% \textbf{model} & \textbf{musk} & \textbf{optdigits} & \textbf{pima} & \textbf{satimage-2} & \textbf{smtp} & \textbf{vertebral} & \textbf{vowels} &  \textbf{wbc} & \textbf{wine} \\
% \hline
%         CHAODA-fast &        858.15 &            3279.67 &         12.36 &              358.49 &        272.61 &               1.88 &          123.34 &          5.26 &          0.39 \\
% \hline
%         CHAODA &        982.25 &            5865.12 &         13.50 &              611.14 &        451.74 &               2.21 &          181.91 &          6.10 &          0.40 \\
% \hline
%                 ABOD &          4.39 &               6.81 &          0.26 &                3.20 &         18.00 &      \textbf{0.08} &            0.51 &          0.14 & \textbf{0.04} \\
% \hline
%         AutoEncoder &         23.40 &              24.29 &          4.65 &               23.73 &        195.61 &               2.85 &            6.98 &          4.70 &          3.31 \\
% \hline
%                 CBLOF &          0.27 &               0.44 & \textbf{0.08} &                0.31 &          0.65 &      \textbf{0.06} &   \textbf{0.10} & \textbf{0.08} & \textbf{0.05} \\
% \hline
%                 COF &         45.60 &             119.57 &          2.40 &              141.84 &      21283.99 &               0.28 &            8.06 &          0.68 &          0.11 \\
% \hline
%                 HBOS & \textbf{0.07} &      \textbf{0.04} & \textbf{0.00} &       \textbf{0.02} & \textbf{0.01} &      \textbf{0.00} &   \textbf{0.01} & \textbf{0.01} & \textbf{0.00} \\
% \hline
%         IFOREST &          0.93 &               0.92 &          0.35 &                0.79 &          3.90 &               0.30 &            0.40 &          0.33 &          0.29 \\
% \hline
%                 KNN &          3.50 &               5.53 & \textbf{0.09} &                1.84 &          6.83 &      \textbf{0.03} &            0.18 & \textbf{0.05} & \textbf{0.01} \\
% \hline
%                 LMDD &        375.11 &             421.22 &          2.86 &              302.94 &       3872.57 &               0.59 &            9.47 &          1.91 &          0.33 \\
% \hline
%                 LOCI &   \textit{TO} &        \textit{TO} &   \textit{TO} &         \textit{TO} &   \textit{TO} &              35.97 &     \textit{TO} &        154.49 &          6.84 \\
% \hline
%                 LODA & \textbf{0.07} &      \textbf{0.10} & \textbf{0.04} &       \textbf{0.10} &          0.73 &      \textbf{0.03} &   \textbf{0.05} & \textbf{0.04} & \textbf{0.03} \\
% \hline
%                 LOF &          3.59 &               5.49 & \textbf{0.02} &                1.50 &          1.36 &      \textbf{0.00} &   \textbf{0.06} & \textbf{0.01} & \textbf{0.00} \\
% \hline
%                 MCD &         84.74 &               7.24 &          0.69 &                6.83 &         13.78 &      \textbf{0.05} &            0.81 &          0.11 & \textbf{0.05} \\
% \hline
%         MOGAAL &        267.93 &             415.02 &         40.89 &              463.18 &   \textit{TO} &              39.86 &           80.35 &         40.65 &         38.10 \\
% \hline
%                 OCSVM &          2.51 &               3.71 & \textbf{0.03} &                3.14 &        252.52 &      \textbf{0.00} &            0.11 & \textbf{0.01} & \textbf{0.00} \\
% \hline
%                 SOD &         56.94 &             131.49 &          2.48 &              210.39 &   \textit{TO} &               0.54 &           13.92 &          1.03 &          0.19 \\
% \hline
%         SOGAAL &         29.55 &              44.72 &          5.20 &               48.79 &        592.27 &               4.69 &           10.38 &          4.76 &          4.01 \\
% \hline
%                 SOS &          9.37 &              26.09 &          1.01 &               34.96 &   \textit{TO} &               0.21 &            2.81 &          0.36 & \textbf{0.10} \\
% \hline
%                 VAE &         30.38 &              30.58 &          5.67 &               30.41 &        176.84 &               4.03 &           10.46 &          5.33 &          4.62 \\
% \hline
% \end{tabular}
% \end{table*}


% TODO: Add the following tables to the supplementary materials.


% \subsection{New CHAODA Results}
% \label{subsec:results:new-chaoda-results}

% \begin{table*}[!b]
%     \renewcommand{\arraystretch}{1.25}
%     \caption{New CHAODA Results. Set 1}
%     \label{table:results:new-chaoda-1}
%     \centering
%     \begin{tabular}{|c|c|c|c|c|c|c|c|c|c|}
%         \hline
%         \textbf{voting} & \textbf{normed} & \textbf{\textbf{annthyroid}} & \textbf{\textbf{arrhythmia}} & \textbf{\textbf{breastw}} & \textbf{\textbf{cardio}} & \textbf{\textbf{cover}} & \textbf{\textbf{glass}} & \textbf{\textbf{http}} & \textbf{\textbf{ionosphere}} \\
%         \hline
%                    mean &            None &                \textbf{0.64} &                \textbf{0.77} &             \textbf{0.97} &            \textbf{0.81} &                    0.71 &                    0.70 &          \textbf{1.00} &                \textbf{0.88} \\
%         \hline
%                 product &            None &                         0.62 &                \textbf{0.77} &                      0.92 &                     0.76 &                    0.71 &                    0.63 &          \textbf{1.00} &                         0.80 \\
%         \hline
%                  median &            None &                         0.51 &                         0.57 &                      0.95 &                     0.55 &                    0.49 &                    0.53 &          \textbf{1.00} &                         0.60 \\
%         \hline
%                     min &            None &                         0.61 &                         0.73 &             \textbf{0.98} &            \textbf{0.79} &                    0.71 &                    0.69 &          \textbf{1.00} &                \textbf{0.87} \\
%         \hline
%                     max &            None &                         0.62 &                \textbf{0.75} &             \textbf{0.96} &            \textbf{0.81} &                    0.73 &           \textbf{0.73} &          \textbf{1.00} &                \textbf{0.88} \\
%         \hline
%                     p25 &            None &                \textbf{0.65} &                \textbf{0.75} &             \textbf{0.96} &            \textbf{0.80} &                    0.45 &                    0.70 &          \textbf{1.00} &                \textbf{0.87} \\
%         \hline
%                     p75 &            None &                         0.51 &                         0.74 &             \textbf{0.98} &                     0.73 &           \textbf{0.77} &           \textbf{0.75} &          \textbf{1.00} &                         0.61 \\
%         \hline
%                    mean &          linear &                         0.58 &                \textbf{0.75} &             \textbf{0.98} &            \textbf{0.79} &                    0.74 &           \textbf{0.73} &          \textbf{1.00} &                         0.85 \\
%         \hline
%                 product &          linear &                         0.61 &                         0.73 &             \textbf{0.98} &            \textbf{0.79} &                    0.71 &                    0.69 &          \textbf{1.00} &                \textbf{0.87} \\
%         \hline
%                  median &          linear &                         0.62 &                \textbf{0.75} &             \textbf{0.96} &            \textbf{0.81} &                    0.73 &           \textbf{0.73} &          \textbf{1.00} &                \textbf{0.88} \\
%         \hline
%                     min &          linear &                \textbf{0.65} &                \textbf{0.75} &             \textbf{0.96} &            \textbf{0.80} &                    0.45 &                    0.70 &          \textbf{1.00} &                \textbf{0.87} \\
%         \hline
%                     max &          linear &                         0.51 &                         0.74 &             \textbf{0.98} &                     0.73 &           \textbf{0.77} &           \textbf{0.75} &          \textbf{1.00} &                         0.61 \\
%         \hline
%                     p25 &          linear &                \textbf{0.64} &                \textbf{0.77} &             \textbf{0.97} &            \textbf{0.81} &                    0.71 &                    0.70 &          \textbf{1.00} &                \textbf{0.88} \\
%         \hline
%                     p75 &          linear &                         0.62 &                \textbf{0.77} &                      0.92 &                     0.76 &                    0.71 &                    0.63 &          \textbf{1.00} &                         0.80 \\
%         \hline
%                    mean &        gaussian &                         0.62 &                \textbf{0.75} &             \textbf{0.96} &            \textbf{0.81} &                    0.73 &           \textbf{0.73} &          \textbf{1.00} &                \textbf{0.88} \\
%         \hline
%                 product &        gaussian &                \textbf{0.65} &                \textbf{0.75} &             \textbf{0.96} &            \textbf{0.80} &                    0.45 &                    0.70 &          \textbf{1.00} &                \textbf{0.87} \\
%         \hline
%                  median &        gaussian &                         0.51 &                         0.70 &             \textbf{0.98} &                     0.73 &           \textbf{0.77} &           \textbf{0.75} &          \textbf{1.00} &                         0.61 \\
%         \hline
%                     min &        gaussian &                \textbf{0.64} &                \textbf{0.77} &             \textbf{0.97} &            \textbf{0.81} &                    0.71 &                    0.70 &          \textbf{1.00} &                \textbf{0.88} \\
%         \hline
%                     max &        gaussian &                         0.62 &                \textbf{0.77} &                      0.92 &                     0.76 &                    0.71 &                    0.63 &          \textbf{1.00} &                         0.80 \\
%         \hline
%                     p25 &        gaussian &                         0.54 &                         0.64 &                      0.50 &                     0.73 &                    0.50 &                    0.56 &          \textbf{1.00} &                         0.70 \\
%         \hline
%                     p75 &        gaussian &                         0.61 &                         0.73 &             \textbf{0.98} &            \textbf{0.79} &                    0.71 &                    0.69 &          \textbf{1.00} &                \textbf{0.87} \\
%         \hline
%                    mean &         sigmoid &                         0.51 &                         0.74 &             \textbf{0.98} &                     0.73 &           \textbf{0.77} &           \textbf{0.75} &          \textbf{1.00} &                         0.61 \\
%         \hline
%                 product &         sigmoid &                \textbf{0.64} &                \textbf{0.77} &             \textbf{0.97} &            \textbf{0.81} &                    0.71 &                    0.70 &          \textbf{1.00} &                \textbf{0.88} \\
%         \hline
%                  median &         sigmoid &                         0.62 &                \textbf{0.77} &                      0.92 &                     0.76 &                    0.71 &                    0.63 &          \textbf{1.00} &                         0.80 \\
%         \hline
%                     min &         sigmoid &                         0.54 &                         0.66 &                      0.50 &                     0.73 &                    0.50 &                    0.56 &          \textbf{1.00} &                         0.69 \\
%         \hline
%                     max &         sigmoid &                         0.61 &                         0.73 &             \textbf{0.98} &            \textbf{0.79} &                    0.71 &                    0.69 &          \textbf{1.00} &                \textbf{0.87} \\
%         \hline
%                     p25 &         sigmoid &                         0.62 &                \textbf{0.75} &             \textbf{0.96} &            \textbf{0.81} &                    0.73 &           \textbf{0.73} &          \textbf{1.00} &                \textbf{0.88} \\
%         \hline
%                     p75 &         sigmoid &                \textbf{0.65} &                \textbf{0.75} &             \textbf{0.96} &            \textbf{0.80} &                    0.45 &                    0.70 &          \textbf{1.00} &                \textbf{0.87} \\
%         \hline
%     \end{tabular}
% \end{table*}


% \begin{table*}[!b]
%     \renewcommand{\arraystretch}{1.25}
%     \caption{New CHAODA Results. Set 2}
%     \label{table:results:new-chaoda-2}
%     \centering
%     \begin{tabular}{|c|c|c|c|c|c|c|c|c|c|}
%         \hline
%         \textbf{voting} & \textbf{normed} & \textbf{\textbf{lympho}} & \textbf{\textbf{mammography}} & \textbf{\textbf{mnist}} & \textbf{\textbf{musk}} & \textbf{\textbf{optdigits}} & \textbf{\textbf{pendigits}} & \textbf{\textbf{pima}} & \textbf{\textbf{satellite}} \\
%         \hline
%                    mean &            None &            \textbf{0.98} &                 \textbf{0.85} &           \textbf{0.78} &          \textbf{1.00} &               \textbf{0.96} &               \textbf{0.94} &                   0.63 &               \textbf{0.79} \\
%         \hline
%                 product &            None &                     0.92 &                          0.80 &                    0.71 &          \textbf{1.00} &                        0.88 &                        0.88 &                   0.60 &                        0.76 \\
%         \hline
%                  median &            None &                     0.91 &                          0.52 &                    0.56 &          \textbf{1.00} &                        0.81 &                        0.69 &                   0.52 &                        0.67 \\
%         \hline
%                     min &            None &            \textbf{0.97} &                          0.83 &                    0.76 &          \textbf{1.00} &               \textbf{0.95} &               \textbf{0.95} &                   0.64 &               \textbf{0.79} \\
%         \hline
%                     max &            None &            \textbf{0.98} &                 \textbf{0.86} &                    0.75 &          \textbf{1.00} &                        0.88 &               \textbf{0.94} &                   0.63 &               \textbf{0.79} \\
%         \hline
%                     p25 &            None &                     0.95 &                 \textbf{0.85} &                    0.66 &          \textbf{1.00} &                        0.91 &               \textbf{0.94} &                   0.62 &               \textbf{0.77} \\
%         \hline
%                     p75 &            None &                     0.94 &                          0.83 &                    0.71 &          \textbf{0.99} &                        0.82 &                        0.64 &                   0.60 &                        0.72 \\
%         \hline
%                    mean &          linear &            \textbf{0.98} &                 \textbf{0.85} &           \textbf{0.79} &          \textbf{1.00} &                        0.93 &                        0.77 &          \textbf{0.67} &               \textbf{0.79} \\
%         \hline
%                 product &          linear &            \textbf{0.97} &                          0.83 &                    0.76 &          \textbf{1.00} &               \textbf{0.95} &               \textbf{0.95} &                   0.64 &               \textbf{0.79} \\
%         \hline
%                  median &          linear &            \textbf{0.98} &                 \textbf{0.86} &                    0.75 &          \textbf{1.00} &                        0.88 &               \textbf{0.94} &                   0.63 &               \textbf{0.79} \\
%         \hline
%                     min &          linear &                     0.95 &                 \textbf{0.85} &                    0.66 &          \textbf{1.00} &                        0.91 &               \textbf{0.94} &                   0.62 &               \textbf{0.77} \\
%         \hline
%                     max &          linear &                     0.94 &                          0.83 &                    0.71 &          \textbf{0.99} &                        0.82 &                        0.64 &                   0.60 &                        0.72 \\
%         \hline
%                     p25 &          linear &            \textbf{0.98} &                 \textbf{0.85} &           \textbf{0.78} &          \textbf{1.00} &               \textbf{0.96} &               \textbf{0.94} &                   0.63 &               \textbf{0.79} \\
%         \hline
%                     p75 &          linear &                     0.92 &                          0.80 &                    0.71 &          \textbf{1.00} &                        0.88 &                        0.88 &                   0.60 &                        0.76 \\
%         \hline
%                    mean &        gaussian &            \textbf{0.98} &                 \textbf{0.86} &                    0.75 &          \textbf{1.00} &                        0.88 &               \textbf{0.94} &                   0.63 &               \textbf{0.79} \\
%         \hline
%                 product &        gaussian &                     0.95 &                 \textbf{0.85} &                    0.66 &          \textbf{1.00} &                        0.91 &               \textbf{0.94} &                   0.62 &               \textbf{0.77} \\
%         \hline
%                  median &        gaussian &                     0.94 &                          0.83 &                    0.71 &          \textbf{0.99} &                        0.82 &                        0.65 &                   0.60 &                        0.72 \\
%         \hline
%                     min &        gaussian &            \textbf{0.98} &                 \textbf{0.85} &           \textbf{0.78} &          \textbf{1.00} &               \textbf{0.96} &               \textbf{0.94} &                   0.63 &               \textbf{0.79} \\
%         \hline
%                     max &        gaussian &                     0.92 &                          0.80 &                    0.71 &          \textbf{1.00} &                        0.88 &                        0.88 &                   0.60 &                        0.76 \\
%         \hline
%                     p25 &        gaussian &            \textbf{0.98} &                          0.60 &                    0.69 &          \textbf{1.00} &                        0.50 &                        0.50 &                   0.57 &                        0.50 \\
%         \hline
%                     p75 &        gaussian &            \textbf{0.97} &                          0.83 &                    0.76 &          \textbf{1.00} &               \textbf{0.95} &               \textbf{0.95} &                   0.64 &               \textbf{0.79} \\
%         \hline
%                    mean &         sigmoid &                     0.94 &                          0.83 &                    0.71 &          \textbf{0.99} &                        0.82 &                        0.64 &                   0.60 &                        0.72 \\
%         \hline
%                 product &         sigmoid &            \textbf{0.98} &                 \textbf{0.85} &           \textbf{0.78} &          \textbf{1.00} &               \textbf{0.96} &               \textbf{0.94} &                   0.63 &               \textbf{0.79} \\
%         \hline
%                  median &         sigmoid &                     0.92 &                          0.80 &                    0.71 &          \textbf{1.00} &                        0.88 &                        0.88 &                   0.60 &                        0.76 \\
%         \hline
%                     min &         sigmoid &            \textbf{0.98} &                          0.60 &                    0.70 &          \textbf{1.00} &                        0.50 &                        0.50 &                   0.56 &                        0.50 \\
%         \hline
%                     max &         sigmoid &            \textbf{0.97} &                          0.83 &                    0.76 &          \textbf{1.00} &               \textbf{0.95} &               \textbf{0.95} &                   0.64 &               \textbf{0.79} \\
%         \hline
%                     p25 &         sigmoid &            \textbf{0.98} &                 \textbf{0.86} &                    0.75 &          \textbf{1.00} &                        0.88 &               \textbf{0.94} &                   0.63 &               \textbf{0.79} \\
%         \hline
%                     p75 &         sigmoid &                     0.95 &                 \textbf{0.85} &                    0.66 &          \textbf{1.00} &                        0.91 &               \textbf{0.94} &                   0.62 &               \textbf{0.77} \\
%         \hline
%     \end{tabular}              
% \end{table*}


% \begin{table*}[!b]
%     \renewcommand{\arraystretch}{1.25}
%     \caption{New CHAODA Results. Set 3}
%     \label{table:results:new-chaoda-3}
%     \centering
%     \begin{tabular}{|c|c|c|c|c|c|c|c|c|c|}
%         \hline
%         \textbf{voting} & \textbf{normed} & \textbf{\textbf{satimage-2}} & \textbf{\textbf{shuttle}} & \textbf{\textbf{smtp}} & \textbf{\textbf{thyroid}} & \textbf{\textbf{vertebral}} & \textbf{\textbf{vowels}} & \textbf{\textbf{wbc}} & \textbf{\textbf{wine}} \\
%         \hline
%                     mean &            None &                \textbf{1.00} &             \textbf{0.51} &                   0.92 &             \textbf{0.89} &                        0.29 &                     0.71 &         \textbf{0.97} &          \textbf{0.99} \\
%         \hline
%                 product &            None &                         0.95 &             \textbf{0.51} &                   0.45 &                      0.75 &                        0.36 &                     0.59 &         \textbf{0.97} &          \textbf{0.98} \\
%         \hline
%                     median &            None &                         0.93 &             \textbf{0.50} &                   0.84 &                      0.56 &               \textbf{0.46} &                     0.52 &                  0.82 &          \textbf{0.99} \\
%         \hline
%                     min &            None &                \textbf{0.98} &             \textbf{0.51} &                   0.91 &                      0.83 &                        0.32 &                     0.67 &                  0.95 &          \textbf{0.98} \\
%         \hline
%                     max &            None &                \textbf{0.99} &             \textbf{0.51} &          \textbf{0.95} &             \textbf{0.90} &                        0.30 &                     0.70 &         \textbf{0.97} &          \textbf{1.00} \\
%         \hline
%                     p25 &            None &                \textbf{0.99} &             \textbf{0.51} &                   0.87 &             \textbf{0.90} &                        0.29 &                     0.69 &         \textbf{0.98} &          \textbf{1.00} \\
%         \hline
%                     p75 &            None &                         0.96 &             \textbf{0.51} &                   0.87 &                      0.62 &               \textbf{0.47} &                     0.72 &                  0.89 &          \textbf{0.99} \\
%         \hline
%                     mean &          linear &                \textbf{0.99} &             \textbf{0.51} &                   0.91 &             \textbf{0.88} &                        0.36 &            \textbf{0.79} &                  0.95 &          \textbf{0.99} \\
%         \hline
%                 product &          linear &                \textbf{0.98} &             \textbf{0.51} &                   0.91 &                      0.83 &                        0.32 &                     0.67 &                  0.95 &          \textbf{0.98} \\
%         \hline
%                     median &          linear &                \textbf{0.99} &             \textbf{0.51} &          \textbf{0.95} &             \textbf{0.90} &                        0.30 &                     0.70 &         \textbf{0.97} &          \textbf{1.00} \\
%         \hline
%                     min &          linear &                \textbf{0.99} &             \textbf{0.51} &                   0.87 &             \textbf{0.90} &                        0.29 &                     0.69 &         \textbf{0.98} &          \textbf{1.00} \\
%         \hline
%                     max &          linear &                         0.96 &             \textbf{0.51} &                   0.87 &                      0.62 &               \textbf{0.47} &                     0.72 &                  0.89 &          \textbf{0.99} \\
%         \hline
%                     p25 &          linear &                \textbf{1.00} &             \textbf{0.51} &                   0.92 &             \textbf{0.89} &                        0.29 &                     0.71 &         \textbf{0.97} &          \textbf{0.99} \\
%         \hline
%                     p75 &          linear &                         0.95 &             \textbf{0.51} &                   0.45 &                      0.75 &                        0.36 &                     0.59 &         \textbf{0.97} &          \textbf{0.98} \\
%         \hline
%                     mean &        gaussian &                \textbf{0.99} &             \textbf{0.51} &          \textbf{0.95} &             \textbf{0.90} &                        0.30 &                     0.70 &         \textbf{0.97} &          \textbf{1.00} \\
%         \hline
%                 product &        gaussian &                \textbf{0.99} &             \textbf{0.51} &                   0.87 &             \textbf{0.90} &                        0.29 &                     0.69 &         \textbf{0.98} &          \textbf{1.00} \\
%         \hline
%                     median &        gaussian &                         0.96 &             \textbf{0.51} &                   0.87 &                      0.62 &               \textbf{0.47} &                     0.72 &                  0.89 &          \textbf{0.99} \\
%         \hline
%                     min &        gaussian &                \textbf{1.00} &             \textbf{0.51} &                   0.92 &             \textbf{0.89} &                        0.29 &                     0.71 &         \textbf{0.97} &          \textbf{0.99} \\
%         \hline
%                     max &        gaussian &                         0.95 &             \textbf{0.51} &                   0.45 &                      0.75 &                        0.36 &                     0.59 &         \textbf{0.97} &          \textbf{0.98} \\
%         \hline
%                     p25 &        gaussian &                \textbf{0.98} &             \textbf{0.50} &                   0.50 &                      0.63 &               \textbf{0.46} &                     0.50 &                  0.91 &          \textbf{0.99} \\
%         \hline
%                     p75 &        gaussian &                \textbf{0.98} &             \textbf{0.51} &                   0.91 &                      0.83 &                        0.32 &                     0.67 &                  0.95 &          \textbf{0.98} \\
%         \hline
%                     mean &         sigmoid &                         0.96 &             \textbf{0.51} &                   0.87 &                      0.62 &               \textbf{0.47} &                     0.72 &                  0.89 &          \textbf{0.99} \\
%         \hline
%                 product &         sigmoid &                \textbf{1.00} &             \textbf{0.51} &                   0.92 &             \textbf{0.89} &                        0.29 &                     0.71 &         \textbf{0.97} &          \textbf{0.99} \\
%         \hline
%                     median &         sigmoid &                         0.95 &             \textbf{0.51} &                   0.45 &                      0.75 &                        0.36 &                     0.59 &         \textbf{0.97} &          \textbf{0.98} \\
%         \hline
%                     min &         sigmoid &                \textbf{0.98} &             \textbf{0.50} &                   0.50 &                      0.62 &               \textbf{0.46} &                     0.50 &                  0.91 &          \textbf{0.99} \\
%         \hline
%                     max &         sigmoid &                \textbf{0.98} &             \textbf{0.51} &                   0.91 &                      0.83 &                        0.32 &                     0.67 &                  0.95 &          \textbf{0.98} \\
%         \hline
%                     p25 &         sigmoid &                \textbf{0.99} &             \textbf{0.51} &          \textbf{0.95} &             \textbf{0.90} &                        0.30 &                     0.70 &         \textbf{0.97} &          \textbf{1.00} \\
%         \hline
%                     p75 &         sigmoid &                \textbf{0.99} &             \textbf{0.51} &                   0.87 &             \textbf{0.90} &                        0.29 &                     0.69 &         \textbf{0.98} &          \textbf{1.00} \\
%         \hline
%     \end{tabular}   
% \end{table*}

\end{document}
\end{document}
